\documentclass[hidelinks, 12pt]{article} 
\usepackage{geometry}   \geometry{letterpaper}  
\usepackage{color}

\usepackage[parfill]{parskip}   
\usepackage{graphicx}	
	
\addtolength{\oddsidemargin}{-0.8in}
\addtolength{\evensidemargin}{-0.8in}
\addtolength{\textwidth}{1.6in}
\addtolength{\topmargin}{-.5in}
\addtolength{\textheight}{1in}	
	
\usepackage{amssymb}
\usepackage{amsmath}
\usepackage{commath}

\DeclareMathOperator{\Id}{Id}
\DeclareMathOperator{\mult}{mult}
\DeclareMathOperator{\ord}{ord}

\usepackage{amsthm}

\newtheoremstyle{mydefstyle}
    {6pt}
    {3pt}
    {}
    {}
    {\bfseries}
    {}
    { }
    {\thmname{#1} \normalfont{\thmnote{(#3)}\addcontentsline{toc}{subsubsection}{\bf{#1} \normalfont{(#3)}}}}
    
\theoremstyle{mydefstyle}
\newtheorem{definition}{Definition}
\newtheorem{example}{Example}

\newtheoremstyle{mythmstyle}
    {6pt}
    {3pt}
    {}
    {}
    {\bfseries}
    {}
    { }
    {\thmname{#1} \thmnumber{#2} \normalfont{\thmnote{(#3)}\addcontentsline{toc}{subsubsection}{\bf{#1 #2} \normalfont{(#3)}}}}
    
\theoremstyle{mythmstyle} 
\newcounter{prop}
\newtheorem{proposition}[prop]{Proposition}
\newtheorem{theorem}[prop]{Theorem}
\newtheorem{corollary}[prop]{Corollary}
\newtheorem{lemma}[prop]{Lemma}

\usepackage{tikz-cd}
\usepackage{bm}
\usepackage[shortlabels]{enumitem}












\title{Algebraic Curves and Riemann Surfaces}
\date{}

\begin{document}
%\maketitle
\pagecolor{white}
%\tableofcontents

\textbf{I.1 Complex Charts and Complex Structures}

Recall the classification of compact orientable 2-manifolds; each of these is a $g$-holed torus for some unique integer $g \ge 0$. This integer $g$ is called the topological genus of the compact Riemann surface.

\textbf{II.1 Functions on Riemann Surfaces}

Let $f$ be meromorphic at $p$, whose Laurent series in a local coordinate $z$ is $\sum_n c_n(z-z_0)^n$. The order of $f$ at $p$, denoted by $\ord_p(f)$, is the minimum exponent actually appearing (with nonzero coefficient in the Laurent series:
\begin{gather*}
\ord_p(f) = \min\{n \vert c_n \ne 0 \}
\end{gather*}

\textbf{II.4 Global Properties of Holomorphic Maps}

Let $F : X \to Y$ be a nonconstant holomorphic map between compact Riemann surfaces. 

The multiplicity of $F$ at $p$, denoted $\mult_p(F)$, is the unique integer $m$ such that there are local coordinates near $p$ and $F(p)$ with $F$ having the form $z \mapsto z^m$.

\textbf{Proposition} For each $y \in Y$, define $d_y(F)$ to be the sum of the multiplicities of $F$ at the points of $X$ mapping to $y$:
\begin{gather*}
d_y(F) = \sum_{p \in F^{-1}(y)} \mult_p(F)
\end{gather*}
Then $d_y(F)$ is constant, independent of $y$.

The degree of $F$, denoted $\deg(F)$, is the integer $d_y(F)$ for any $y \in Y$.

\textbf{Theorem} (Hurwitz's formula)
\begin{gather*}
2g(X) - 2 = \deg(F)(2g(Y) - 2) + \sum_{p \in X} (\mult_p(F) - 1)
\end{gather*}

\textbf{Problem} Let $X$ be the projective plane curve of degree $d$ defined by the homogeneous polynomial $F(x, y, z) = x^d + y^d + z^d$. This curve is called the Fermat curve of degree $d$. Let $\pi : X \to \mathbb{P}^1$ be given by $\pi[x:y:z] = [x:y]$. Use Hurwitz's formula to show that the genus of the Fermat curve is
\begin{gather*}
g(X) = \frac{(d-1)(d-2)}{2}
\end{gather*}

\end{document}