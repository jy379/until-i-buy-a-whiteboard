\documentclass[hidelinks, 12pt]{article} 
\usepackage{geometry}   \geometry{letterpaper}  
\usepackage{color}

\usepackage[parfill]{parskip}   
\usepackage{graphicx}	
	
\addtolength{\oddsidemargin}{-0.8in}
\addtolength{\evensidemargin}{-0.8in}
\addtolength{\textwidth}{1.6in}
\addtolength{\topmargin}{-.5in}
\addtolength{\textheight}{1in}	
	
\usepackage{amssymb}
\usepackage{amsmath}
\usepackage{commath}

\DeclareMathOperator{\Id}{Id}

\usepackage{amsthm}

\newtheoremstyle{mydefstyle}
    {6pt}
    {3pt}
    {}
    {}
    {\bfseries}
    {}
    { }
    {\thmname{#1} \normalfont{\thmnote{(#3)}\addcontentsline{toc}{subsubsection}{\bf{#1} \normalfont{(#3)}}}}
    
\theoremstyle{mydefstyle}
\newtheorem{definition}{Definition}
\newtheorem{example}{Example}

\newtheoremstyle{mythmstyle}
    {6pt}
    {3pt}
    {}
    {}
    {\bfseries}
    {}
    { }
    {\thmname{#1} \thmnumber{#2} \normalfont{\thmnote{(#3)}\addcontentsline{toc}{subsubsection}{\bf{#1 #2} \normalfont{(#3)}}}}
    
\theoremstyle{mythmstyle} 
\newcounter{prop}
\newtheorem{proposition}[prop]{Proposition}
\newtheorem{theorem}[prop]{Theorem}
\newtheorem{corollary}[prop]{Corollary}
\newtheorem{lemma}[prop]{Lemma}

\usepackage{tikz-cd}
\usepackage{bm}
\usepackage[shortlabels]{enumitem}












\title{Winter Break 2024-25}
\date{}

\begin{document}
%\maketitle
\pagecolor{white}
%\tableofcontents

\textbf{1B Markov Chains}

\textbf{1B Statistics}

\textbf{Part 2 Probability \& Measure}

The relationships between various modes of convergence: almost sure, in probability, in $L^p$. Strong law of large numbers and central limit theorem. 

\textbf{Part 2 Applied Probability}

Review statements and proofs of the basic theorems on queues and the starting point of the story for Poisson processes. Bauerschmidt's notes. 

\textbf{Part 3 Advanced Probability}

Would really like to get hold of an interesting almost current concrete research problem to give direction to the theory. Martingale convergence theorems and basic secret examples of stochastic integrals and Feynman-Kac. 

\textbf{Part 3 Stochastic Calculus}

Would really like to get hold of an interesting almost current concrete research problem to give direction to the theory. 

\textbf{Number Theory}

Greg is taking this in the spring semester though I probably can't fit it into my schedule and the registration is already full last I checked anyways. My main interest in this is to find motivation for commutative algebra from a source other than algebraic geometry, and also to see what the algebraic number theorists have to say about solving diophantine equations, which seems to be a basic and interesting problem, as introduced by Shafarevich. 

Not sure what books or notes to start with. Could look into the Cambridge notes and 

\textbf{Algebraic Geometry}

Want to understand (moduli spaces of) J-holomorphic curves in symplectic geometry. This was invented by Gromov and apparently inspired by pre-existing theory of holomorphic curves in algebraic geometry. Sounds to me it is basically about counting how many lines or curves of a particular degree there are in an algebraic variety. Like ``27 lines on a cubic hypersurface" kind of thing, which is of course the classical starting point of the mirror symmetry business. 

So I should learn about (moduli spaces of) algebraic curves. And probably Riemann surfaces and compare and contrast. ``Curves and their Jacobians" seems to be a thing. Riemann surfaces version is in the Rick Miranda book. 

Prerequisite for reading about curves and their Jacobians is to review and strengthen general basic algebraic geometry e.g. from Shafarevich, Kempf, Mumford. Would like to also take the opportunity to learn about sheaves, sheaf cohomology, and schemes. Mumford's Red Book is probably the desired ``endpoint", though I am not sure how it compares vs Hartshorne. Shafarevich is very nice and concrete but does not talk about sheaves or schemes. Kempf would be a useful stepping stone. 

So probably review Shafarevich for concrete elementary examples, read Kempf for the sheaves, read Mumford for the schemes, then read about moduli spaces of curves and Jacobians of curves. I suddenly recall something about Hilbert polynomial or Hilbert scheme which I should also look into. Of course on the Riemann surfaces side there is also the thing about period integrals and Hodge theory that I have wanted to look into for the longest time ever. 

\textbf{Symplectic Geometry}

Want to understand framework and applications and computations of Morse theory and Floer homology, in particular J-holomorphic curves for the Floer homology and also more generally in symplectic geometry other than Floer homology. As always, Audin \& Damian and the original papers of Floer, but also Abouzaid for the symplectic homology and cohomology and isomorphisms with string or loop homology for the non-compact manifolds.

Other basic concepts: Liouville manifolds, contact manifolds, Legendrian submanifolds, conical ends, Lagrangian torus fibrations over an affine base, almost complex structures which of course reminds one of K{\"a}hler manifolds and Calabi-Yau manifolds, the canonical and anticanonical bundles. 

\textbf{3-Manifolds}

\textbf{Differential Geometry}

Origins of the subject leading to theories of Riemannian metrics, curvature, and connections. Spivak of course, but also PMH Wilson, which might remind me of some questions regarding Lie groups and hence topological and algebraic groups. 

\textbf{Algebraic Topology}

Fundamental class, characteristic classes, computations of homology and cohomology using simplicial and other theories, isomorphisms between these theories with a view towards computations. Spin manifolds. 

\textbf{General brainteasers and problem solving exercises}

Literally to prepare for interviews and online assessments. Would like to solve and create my own collection. Start from quant interview guide books and 1A Numbers \& Sets but also look into olympiad problems and Putnam and other competitions. 

\end{document}