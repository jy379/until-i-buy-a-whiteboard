\documentclass[hidelinks, 12pt]{article} 
\usepackage{geometry}   \geometry{letterpaper}  
\usepackage{color}

\usepackage[parfill]{parskip}   
\usepackage{graphicx}	
	
\addtolength{\oddsidemargin}{-0.8in}
\addtolength{\evensidemargin}{-0.8in}
\addtolength{\textwidth}{1.6in}
\addtolength{\topmargin}{-.5in}
\addtolength{\textheight}{1in}	
	
\usepackage{amssymb}
\usepackage{amsmath}
\usepackage{commath}

\DeclareMathOperator{\codim}{codim}
\DeclareMathOperator{\Id}{Id}
\DeclareMathOperator{\Ind}{Ind}
\DeclareMathOperator{\Var}{Var}

\usepackage{amsthm}
\usepackage{braket}
\usepackage{lscape}

\newtheoremstyle{mydefstyle}
    {6pt}
    {3pt}
    {}
    {}
    {\bfseries}
    {}
    { }
    {\thmname{#1} \normalfont{\thmnote{(#3)}\addcontentsline{toc}{subsubsection}{\bf{#1} \normalfont{(#3)}}}}
    
\theoremstyle{mydefstyle}
\newtheorem{definition}{Definition}
\newtheorem{example}{Example}

\newtheoremstyle{mythmstyle}
    {6pt}
    {3pt}
    {}
    {}
    {\bfseries}
    {}
    { }
    {\thmname{#1} \thmnumber{#2} \normalfont{\thmnote{(#3)}\addcontentsline{toc}{subsubsection}{\bf{#1 #2} \normalfont{(#3)}}}}
    
\theoremstyle{mythmstyle} 
\newcounter{prop}
\newtheorem{proposition}[prop]{Proposition}
\newtheorem{theorem}[prop]{Theorem}
\newtheorem{corollary}[prop]{Corollary}
\newtheorem{lemma}[prop]{Lemma}

\usepackage{tikz-cd}
\usepackage{bm}
\usepackage[shortlabels]{enumitem}












\title{Winter Break 2024-25}
\date{}

\begin{document}
%\maketitle
\pagecolor{white}
%\tableofcontents

\textbf{1B Markov Chains}

Learn the theorems. Do the 2 example sheets. Weber's notes and cross-reference Norris' book. 

%Let $I$ be a countable set, $\{i, j, k, \dots \}$. Each $i \in I$ is called a \textbf{state} and $I$ is called the \textbf{state-space}.
%
%We work in a \textbf{probability space} $(\Omega, \mathcal{F}, P)$. Here $\omega$ is a set of outcomes, $\mathcal{F}$ is a set of subsets of $\Omega$, and for $A \in \mathcal{F}$, $P(A)$ is the probability of $A$. 
%
%The object of our study is a sequence of random variables $X_0, X_1, \dots$ taking values in $I$, whose joint distribution is determined by simple rules. Recall that a random variable $X$ with values in $I$ is a function $X : \Omega \to I$. 
%
%A row vector $\lambda = (\lambda_i : i \in I)$ is called a \textbf{measure} if $\lambda_i \ge 0$ for all $i$.
%
%If $\sum_i \lambda_i = 1$ then it is a \textbf{distribution} (or probability measure). We start with an \textbf{initial distribution} over $I$, specified by $\{\lambda_i : i \in I\}$ such that $0 \le \lambda_i \le 1$ for all $i$ and $\sum_{i \in I} \lambda_i = 1$.
%
%The special case that with probability 1 we start in state $i$ is denoted $\lambda = \delta_i = (0, \dots, 1, \dots, 0)$.

\textbf{Definition 1.2} We say that $(X_n)_{n \ge 0}$ is a \textbf{Markov chain} with initial distribution $\lambda$ and transition matrix $P$ if for all $n \ge 0$ and $i_0, \dots, i_{n+1} \in I$,
\begin{enumerate}
\item $P(X_0 = i_0) = \lambda_{i_0}$
\item $P(X_{n+1} = i_{n+1} \vert X_0 = i_0, \dots, X_n = i_n) = P(X_{n+1} = i_{n+1} \vert X_n = i_n) = p_{i_n i_{n+1}}$
\end{enumerate}

\textbf{Theorem 1.3} $(X_n)_{n \ge 0}$ is $Markov(\lambda, P)$ iff for all $n \ge 0$ and $i_0, \dots, i_n \in I$,
\begin{gather*}
P(X_0 = i_0, \dots, X_n = i_n) = \lambda_{i_0}p_{i_0 i_1} \dots p_{i_{n-1} i_n} \tag{1.1}
\end{gather*}

\textbf{Example Sheet 1 Michaelmas 2020 (Bauerschmidt)}

\textbf{Q1} Let $X = (X_n)_{n \ge 0}$ be a Markov chain. Show that, conditioned on $X_m = i$, $Z = (Z_n)_{n \ge 0}$ given by $Z_n = X_{n+m}$ is a Markov chain with starting state $i$. 

For each $j \in I$,
\begin{gather*}
P(Z_0 = j) = P(X_m = j) = \delta_{ij} = \begin{cases}
1 \quad&\mbox{ if $j = i$} \\
0 \quad&\mbox{ else}
\end{cases}
\end{gather*}

For each $n \ge 0$ and $i_0, \dots, i_{n+1} \in I$,
\begin{align*}
&\hspace{1mm} P(Z_{n+1} = i_{n+1} \vert Z_0 = i_0, \dots, Z_n = i_n) \\
=&\hspace{1mm} P(X_{m+n+1} = i_{n+1} \vert X_m = i_0, \dots, X_{m+n} = i_{n}) \\
=&\hspace{1mm} \frac{P(X_m = i_0, \dots, X_{m+n} = i_n, X_{m+n+1} = i_{n+1})}{P(X_m = i_0, \dots, X_{m+n} = i_n)} \\
=&\hspace{1mm} \frac {\sum_{j_0, \dots, j_{m-1}} P(X_0 = j_0, \dots, X_{m-1} = j_{m-1}, X_m = i_0, \dots, X_{m+n+1} = i_{n+1}) } {\sum_{k_0, \dots, k_{m-1}} P(X_0 = k_0, \dots, X_{m-1} = k_{m-1}, X_m = i_0, \dots, X_{m+n} = i_n)} \\
=&\hspace{1mm} \frac {\sum_{j_0, \dots, j_{m-1}} P(X_{m+n+1} = i_{n+1} \vert X_0 = j_0, \dots, X_{m-1} = j_{m-1}, X_m = i_0, \dots, X_{m+n} = i_n) P(X_0 = j_0, \dots, X_{m-1} = j_{m-1}, X_m = i_0, \dots, X_{m+n} = i_n) } {\sum_{k_0, \dots, k_{m-1}} P(X_0 = k_0, \dots, X_{m-1} = k_{m-1}, X_m = i_0, \dots, X_{m+n} = i_n)} \\
=&\hspace{1mm} \frac {\sum_{j_0, \dots, j_{m-1}} P(X_{m+n+1} = i_{n+1} \vert X_{m+n} = i_n) P(X_0 = j_0, \dots, X_{m-1} = j_{m-1}, X_m = i_0, \dots, X_{m+n} = i_n) } {\sum_{k_0, \dots, k_{m-1}} P(X_0 = k_0, \dots, X_{m-1} = k_{m-1}, X_m = i_0, \dots, X_{m+n} = i_n)} \\
=&\hspace{1mm} P(X_{m+n+1} = i_{n+1} \vert X_{m+n} = i_n) \frac {\sum_{j_0, \dots, j_{m-1}} P(X_0 = j_0, \dots, X_{m-1} = j_{m-1}, X_m = i_0, \dots, X_{m+n} = i_n) } {\sum_{k_0, \dots, k_{m-1}} P(X_0 = k_0, \dots, X_{m-1} = k_{m-1}, X_m = i_0, \dots, X_{m+n} = i_n)} \\
=&\hspace{1mm} P(X_{m+n+1} = i_{n+1} \vert X_{m+n} = i_n) = p_{i_n i_{n+1}} = P(Z_{n+1} = i_{n+1} \vert Z_n = i_n)
\end{align*}

\textbf{Q2} Let $X = (X_n)_{n \ge 0}$ be a sequence of independent random variables. Show that $X$ is a Markov chain. Under what condition is this chain homogeneous?

Since $X_0, X_1, \dots, X_n$ are independent, for all $i_0, \dots, i_{n+1} \in I$,
\begin{align*}
P(X_{n+1} = i_{n+1} \vert X_0 = i_0, \dots, X_n = i_n)
&= \frac{P(X_0 = i_0, \dots, X_n = i_n, X_{n+1} = i_{n+1})}{P(X_0 = i_0, \dots, X_n = i_n)} \\
&= \frac{P(X_0 = i_0) \dots P(X_n = i_n) P(X_{n+1} = i_{n+1})}{P(X_0 = i_0) \dots P(X_n = i_n)} \\
&= \frac{P(X_n = i_n) P(X_{n+1} = i_{n+1})}{P(X_n = i_n)} \\
&= \frac{P(X_n = i_n, X_{n+1} = i_{n+1})}{P(X_n = i_n)} \\
&= P(X_{n+1} = i_{n+1} \vert X_n = i_n) = P(X_{n+1} = i_{n+1})
\end{align*}
so $X$ is a Markov chain. It is homogeneous iff the $X_i$ are identically distributed. 

\textbf{Q3} Let $X = (X_n)_{n \ge 0}$ be a sequence of fair coin tosses (with the two possible outcomes interpreted as 0 and 1) and set $M_n = \max_{k \le n} X_k$. Show that $(M_n)_{n \ge 0}$ is a Markov chain and find the transition probabilities. 

\begin{gather*}
P(M_{n+1} = 0 \vert M_0 = i_0, \dots, M_n = i_n)
= \begin{cases}
1/2 \quad&\mbox{ if $i_0 = \dots = i_n = 0$} \\
0 \quad&\mbox{ else}
\end{cases}
\end{gather*}
and
\begin{gather*}
P(M_{n+1} = 0 \vert M_n = i_n) = \begin{cases}
1/2 \quad&\mbox{ if $i_n = 0$} \\
0 \quad&\mbox{ else}
\end{cases}
\end{gather*}
but $i_0 = \dots = i_n = 0$ iff $i_n = 0$ so the two (conditional) probabilities are the same. 

\begin{gather*}
P(M_{n+1} = 1 \vert M_0 = i_0, \dots, M_n = i_n)
= \begin{cases}
1/2 \quad&\mbox{ if $i_0 = \dots = i_n = 0$} \\
1 \quad&\mbox{ else}
\end{cases}
\end{gather*}
and
\begin{gather*}
P(M_{n+1} = 1 \vert M_n = i_n) = \begin{cases}
1/2 \quad&\mbox{ if $i_n = 0$} \\
1 \quad&\mbox{ else}
\end{cases}
\end{gather*}
but $i_0 = \dots = i_n = 0$ iff $i_n = 0$ so the two (conditional) probabilities are the same. 

In particular, $p_{01} = p_{00} = 1/2$ and $p_{11} = 1$ and $p_{10} = 0$. 

\textbf{Q4} 

\textbf{1B Statistics}

Do the 3 example sheets. Weber's notes. 

\textbf{Statistics} is a collection of procedures and principles for gaining and processing information in order to make decisions when faced with uncertainty. Two of its principal concerns are \textbf{parameter estimation} and \textbf{hypothesis testing}. 

\textbf{Data} refers to a collection of numbers or other pieces of information to which meaning has been attached. The numbers 1.1, 3.0, 6.5 are not necessarily data. They become so when we are told that they are the muscle weight gains in kg of three athletes who have been trying a new diet. 

In statistics, our data are modelled by a vector of random variables
\begin{gather*}
X = (X_1, X_2, \dots, X_n)
\end{gather*}
where $X_i$ takes values in $\mathbb{Z}$ or $\mathbb{R}$. 

A \textbf{statistic} $T(x)$ is any function of the data. An \textbf{estimator} of a parameter $\theta$ is a function $T = T(X)$ which we use to estimate $\theta$ from an observation of $X$. $T$ is said to be \textbf{unbiased} if
\begin{gather*}
\mathbb{E}(T) = \theta
\end{gather*}
The expectation above is taken over $X$. Once the actual data $x$ is observed, $t = T(x)$ is the \textbf{estimate} of $\theta$ obtained via the estimator $T$. 

Suppose that the random variable $X$ has probability density function $f(x\vert\theta)$. Given the observed value $x$ of $X$, the \textbf{likelihood} of $\theta$ is defined by
\begin{gather*}
lik(\theta) = f(x\vert\theta)
\end{gather*}
Thus we are considering the density as a function of $\theta$, for a fixed $x$. In the case of multiple observations, i.e. when $x = (x_1, \dots, x_n)$ is a vector of observed values of $X_1, \dots, X_n$, we assume unless otherwise stated that $X_1, \dots, X_n$ are IID; in this case $f(x_1, \dots, x_n \vert \theta)$ is the product of the marginals,
\begin{gather*}
lik(\theta) = f(x_1, \dots, x_n \vert \theta) = \prod_{i=1}^n f(x_i \vert \theta)
\end{gather*}

It makes intuitive sense to estimate $\theta$ by whatever value gives the greatest likelihood to the observed data. Thus the \textbf{maximum likelihood estimate} $\hat{\theta}(x)$ of $\theta$ is defined as the value of $\theta$ that maximises the likelihood. Then $\hat{\theta}(X)$ is called the \textbf{maximum likelihood estimator (MLE)} of $\theta$. 

Of course the maximum likelihood estimator need not exist, but in many examples it does. In practice, we usually find the MLE by maximising $\log f(x\vert\theta)$, which is known as the \textbf{loglikelihood}. 

\textbf{1B Optimisation}

Learn the simplex algorithm. Do the 2 example sheets. Weber's notes. 

\textbf{Part 2 Probability \& Measure}

The relationships between various modes of convergence: almost sure, in probability, in $L^p$. Strong law of large numbers and central limit theorem. 

\textbf{Part 2 Applied Probability}

Review statements and proofs of the basic theorems on queues and the starting point of the story for Poisson processes. Bauerschmidt's notes. 

\textbf{Part 3 Advanced Probability}

Would really like to get hold of an interesting almost current concrete research problem to give direction to the theory. Martingale convergence theorems and basic secret examples of stochastic integrals and Feynman-Kac. 

\textbf{Part 3 Stochastic Calculus}

Would really like to get hold of an interesting almost current concrete research problem to give direction to the theory. 

\textbf{Number Theory}

Greg is taking this in the spring semester though I probably can't fit it into my schedule and the registration is already full last I checked anyways. My main interest in this is to find motivation for commutative algebra from a source other than algebraic geometry, and also to see what the algebraic number theorists have to say about solving diophantine equations, which seems to be a basic and interesting problem, as introduced by Shafarevich. 

Not sure what books or notes to start with. Could look into the Cambridge notes and the standard textbook for Greg's course. 

\textbf{Algebraic Geometry}

Want to understand (moduli spaces of) J-holomorphic curves in symplectic geometry. This was invented by Gromov and apparently inspired by pre-existing theory of holomorphic curves in algebraic geometry. Sounds to me it is basically about counting how many lines or curves of a particular degree there are in an algebraic variety. Like ``27 lines on a cubic hypersurface" kind of thing, which is of course the classical starting point of the mirror symmetry business. 

So I should learn about (moduli spaces of) algebraic curves. And probably Riemann surfaces and compare and contrast. ``Curves and their Jacobians" seems to be a thing. Riemann surfaces version is in the Rick Miranda book. 

Prerequisite for reading about curves and their Jacobians is to review and strengthen general basic algebraic geometry e.g. from Shafarevich, Kempf, Mumford. Would like to also take the opportunity to learn about sheaves, sheaf cohomology, and schemes. Mumford's Red Book is probably the desired ``endpoint", though I am not sure how it compares vs Hartshorne. Shafarevich is very nice and concrete but does not talk about sheaves or schemes. Kempf would be a useful stepping stone. 

So probably review Shafarevich for concrete elementary examples, read Kempf for the sheaves, read Mumford for the schemes, then read about moduli spaces of curves and Jacobians of curves. I suddenly recall something about Hilbert polynomial or Hilbert scheme which I should also look into. Of course on the Riemann surfaces side there is also the thing about period integrals and Hodge theory that I have wanted to look into for the longest time ever. 

\textbf{Symplectic Geometry}

Want to understand framework and applications and computations of Morse theory and Floer homology, in particular J-holomorphic curves for the Floer homology and also more generally in symplectic geometry other than Floer homology. As always, Audin \& Damian and the original papers of Floer, but also Abouzaid for the symplectic homology and cohomology and isomorphisms with string or loop homology for the non-compact manifolds.

Other basic concepts: Liouville manifolds, contact manifolds, Legendrian submanifolds, conical ends, Lagrangian torus fibrations over an affine base, almost complex structures which of course reminds one of K{\"a}hler manifolds and Calabi-Yau manifolds, the canonical and anticanonical bundles. 

\textbf{3-Manifolds}

The basic state of knowledge of 3-manifolds. Required for writing the paper with Nathan when the time comes again. Classification of curves on surfaces and the proof that the algorithm works via flip moves or arcslides. 

\textbf{Differential Geometry}

Origins of the subject leading to theories of Riemannian metrics, curvature, and connections. Spivak of course, but also PMH Wilson, which might remind me of some questions regarding Lie groups and hence topological and algebraic groups. 

\textbf{Algebraic Topology}

Fundamental class, characteristic classes, computations of homology and cohomology using simplicial and other theories, isomorphisms between these theories with a view towards computations. Spin manifolds. 

\textbf{General brainteasers and problem solving exercises}

Literally to prepare for interviews and online assessments. Would like to solve and create my own collection. Start from quant interview guide books and 1A Numbers \& Sets but also look into olympiad problems and Putnam and other competitions. 

\newpage

\textbf{Day 1, 27th December 2024, Friday}

\textbf{Q1} 

If $p \in Z(I(X))$ then for every polynomial $f \in k[x_1, \dots, x_n]$ such that $f = 0$ on $X$, have $p \in Z(f)$. So for every Zariski closed subset of $\mathbb{A}^n$ containing $X$, say defined by (the vanishing of) polynomials $f_1, \dots, f_m$, have $p \in Z(f_1, \dots, f_m)$, i.e. $p$ is in every Zariski closed subset of $\mathbb{A}^n$ containing $X$.

Conversely, suppose $p \in \mathbb{A}^n$ is contained in every Zariski closed subset of $\mathbb{A}^n$ containing $X$. Then for any $f \in I(X)$, since $f = 0$ on $X$, have that $X$ is contained in the Zariski closed subset $Z(f)$. Hence $p \in Z(f)$. This shows that $f(p) = 0$ for every $f \in I(X)$ so $p \in Z(I(X))$. 

\textbf{Q2}

Let $X \subseteq \mathbb{A}^n$ be an irreducible affine variety and $U \subseteq X$ a non-empty open set. Let $Y$ be a closed subset of $\mathbb{A}^n$ containing $U$, say $Y = Z(f_1, \dots, f_m)$ for some polynomials $f_1, \dots, f_m$. Suppose there is some $p \in X$ that is not contained in $Y$, so $f_i(p) \ne 0$ for some $i \in \{1, \dots, m\}$, wlog say $f_1(p) \ne 0$. In particular, $p \not\in U$ since $U \subseteq Y$. Write $X_1 = X - U$ and $X_2 = X \cap Z(f_1)$. Clearly these are both closed subsets of $X$, and since $U \subseteq Z(f_1)$, have that $X = X_1 \cup X_2$. But $X_1$ is not the whole of $X$ since $U$ is nonempty, and $X_2$ is not the whole of $X$ since $p \not \in X_2$. This gives $X$ as a union of two proper closed subsets, contradicting the irreducibility of $X$. Thus it is impossible for there to be a closed subset $Y \supseteq U$ but with $p \in X$ and $p \not\in Y$. That is, every closed subset of $\mathbb{A}^n$ containing $U$ must contain all of $X$, i.e. $U$ is dense in $X$. 

Suppose $U$ is not irreducible, so we can write $U = U_1 \cup U_2$ where $U_1, U_2$ are proper closed subsets of $U$. Let $X_1, X_2 \subseteq X$ be closed subsets of $X$ s.t. $U_1 = U \cap X_1$ and $U_2 = U \cap X_2$. Clearly $X_1 \ne X$ since $U_1 \ne U$, and similarly $X_2 \ne X$. But $U = U_1 \cup U_2 = (U \cap X_1) \cup (U \cap X_2) = U \cap (X_1 \cup X_2)$ so $X_1 \cup X_2$ is a closed subset of $X$ containing all of $U$. If $X_1 \cup X_2 = X$, then we would have obtained $X$ as a union of two proper closed subsets, contradicting the irreducibility of $X$. But if $X_1 \cup X_2$ is not all of $X$, then writing $X_3 = X - U$ we definitely have that $X = (X_1 \cup X_2) \cup X_3$ as a union of two proper closed subsets, again contradicting the irreducibility of $X$. Hence it is impossible to write $U$ as a union of proper closed subsets, i.e. $U$ is irreducible.

Suppose $X$ is an irreducible affine variety with at least 2 distinct points $p$ and $q$. If $X$ is Hausdorff, then there are disjoint open subsets $U$ and $V$ of $X$ containing $p$ and $q$ respectively. Write $Y = X - U$ and $Z = X - V$. These are proper closed subsets of $X$ since $p \not\in Y$ and $q \not\in Z$. Furthermore, since $U$ and $V$ are disjoint, have $V \subseteq Y$ and $U \subseteq Z$ so $Y \cup Z = X$. This gives $X$ as a union of proper closed subsets, contradicting the irreducibility of $X$. Hence it is impossible for $X$ to have two distinct points $p$ and $q$ and be Hausdorff. 

\textbf{Q3}

Let $X \supseteq X_1 \supseteq X_2 \supseteq X_3 \supseteq \dots$ be a descending chain of closed subsets in an affine variety $X \subseteq \mathbb{A}^n$. Then $I(X_1) \subseteq I(X_2) \subseteq I(X_3) \subseteq \dots$ is an increasing chain of ideals in $k[x_1, \dots, x_n]$, which is a Noetherian ring, so there is some $N$ s.t. $I(X_n) = I(X_N)$ for all $n \ge N$. Since the $X_i$ are closed subsets, $Z(I(X_i)) = X_i$ by Q1, so $X_n = Z(I(X_n)) = Z(I(X_N)) = X_N$ for all $n \ge N$, i.e. the descending chain of closed subsets is eventually constant.

\textbf{Q4}

The coordinate ring of $Y$ is $k[x, y]/(xy-1)$ but the coordinate ring of $\mathbb{A}^1$ is $k[z]$. An isomorphism between $Y$ and $\mathbb{A}^1$ induces an isomorphism between their coordinate rings. In particular, $x, y \in k[x, y]/(xy-1)$ will be sent under this isomorphism to polynomials $p(z), q(z) \in k[z]$ such that $p(z)q(z) = 1$. But this can only happen if $p(z)$ and $q(z)$ are constants, i.e. if $\mathbb{A}^1$ is mapped to a single point in $Y$ by the isomorphism, which is impossible (because $k$ is algebraically closed so both $Y$ and $\mathbb{A}^1$ have infinitely many points).

As argued above, any morphism $\mathbb{A}^1 \to Y$ induces a pullback of the coordinate functions $x$ and $y$ on $Y$, sending them to polynomial functions $p(z)$ and $q(z)$ on $X$, which must satisfy $p(z)q(z) = 1$, which is only possible if $x$ and $y$ are constants. So the only morphism $\mathbb{A}^1 \to Y$ is the constant map.

Regarding morphisms $Y \to \mathbb{A}^1$, there are the two obvious projections onto the coordinate axes $(x, y) \mapsto x$ and $(x, y) \mapsto y$. More generally, we can project onto any straight line through the origin, $(x, y) \mapsto ax + by$ for some constants $a, b \in k$, and then compose with any morphism $\mathbb{A}^1 \to \mathbb{A}^1$, which is simply a polynomial function in one variable, to obtain $(x, y) \mapsto p(ax + by)$ for any $p(z) \in k[z]$. 

\newpage

\textbf{Day 2, 29th December 2024, Sunday}

\textbf{2.1.c Existence of Pseudo-Gradient Fields}

Pseudo-gradient fields exist for all Morse functions on manifolds. This is, for example, a consequence of the existence of Riemannian metrics, and more exactly, of the existence of Riemannian metrics with a prescribed form on a given subset of the manifold. In any case, it is a simple consequence of the existence of partitions of unity. 

\textbf{Proposition 2.1.5} The stable and unstable manifolds of the critical point $a$ are submanifolds of $V$ that are diffeomorphic to open disks. Moreover, we have
\begin{gather*}
\dim W^u(a) = \codim W^s(a) = \Ind(a)
\end{gather*}

\textbf{Proposition 2.1.6} We suppose that the manifold $V$ is compact. Let $\gamma : \mathbb{R} \to V$ be a trajectory of the pseudo-gradient field $X$. Then there exist critical points $c$ and $d$ of $f$ such that
\begin{gather*}
\lim_{s\to-\infty} \gamma(s) = c \quad\mbox{ and }\quad \lim_{s\to+\infty} \gamma(s) = d
\end{gather*}

\textbf{Theorem 2.2.5 (Smale Theorem)} Let $V$ be a manifold with boundary and let $f$ be a Morse function on $V$ with distinct critical values. We fix Morse charts in the neighbourhood of each critical point of $f$. Let $\Omega$ be the union of these charts and let $X$ be a pseudo-gradient field on $V$ that is transversal to the boundary. Then there exists a pseudo-gradient field $X'$ that is close to $X$ (in the $\mathcal{C}^1$ sense), equals $X$ on $\Omega$ and for which we have
\begin{gather*}
W^s_{X'}(a) \pitchfork W^u_{X'}(b)
\end{gather*}
for all critical points $a, b$ of $f$.

\textbf{Theorem 3.2.2} The space $\overline{\mathcal{L}}(a, b)$ is compact. 

\textbf{Theorem 3.2.7} If $\Ind(a) = \Ind(b) + 2$ then $\overline{\mathcal{L}}(a, b)$ is a compact manifold of dimension 1 with boundary. 

\textbf{Proposition 3.2.8} Let $V$ be a compact manifold, let $f : V \to \mathbb{R}$ be a Morse function ad let $X$ be a pseudo-gradient for $f$ satisfying the Smale property. Let $a, c$ and $b$ be three critical points of indices $k+1$, $k$ and $k-1$ respectively. Let $\lambda_1 \in \mathcal{L}(a, c)$ and $\lambda_2 \in \mathcal{L}(c, b)$. There exists a continuous embedding $\psi$ from an interval $[0, \delta)$ onto a neighbourhood of $(\lambda_1, \lambda_2)$ in $\overline{\mathcal{L}}(a, b)$ that is differentiable on $(0, \delta)$ and satisfies
\begin{gather*}
\psi(0) = (\lambda_1, \lambda_2) \in \overline{\mathcal{L}}(a, b), \qquad \psi(s) \in \mathcal{L}(a, b) \quad\mbox{ for }\quad s \ne 0
\end{gather*}
Moreover, if $(l_n)$ is a sequence in $\mathcal{L}(a, b)$ that tends to $(\lambda_1, \lambda_2)$, then $l_n$ is contained in the image of $\psi$ for $n$ sufficiently large. 

\textbf{Lemma} Let $k$ b an arbitrary field, $f \in k[x, y]$ an irreducible polynomial, and $g \in k[x, y]$ an arbitrary polynomial. If $g$ is not divisible by $f$ then the system of equations $f(x, y) = g(x, y) = 0$ has only a finite number of solutions. 

\textbf{Day 3, 30th December 2024, Monday}

\textbf{2.3 $L^2$ bounded martingales}

\textbf{Definition} Let
\begin{gather*}
M^2 = \{ X : \Omega \times [0, \infty) \to \mathbb{R} : \mbox{ $X$ is a cadlag martingale with $\sup_{t\ge 0} \mathbb{E}X_t^2 < \infty$} \} / \sim \\
M^2_c = \{ X \in M^2 : X(\omega, \cdot) \mbox{ is continuous for every $\omega \in \Omega$} \} / \sim
\end{gather*}
where $\sim$ means that indistinguishable processes are identified. Moreover, set
\begin{gather*}
\norm{X}_{M^2} = \left( \sup_t \mathbb{E}X_t^2 \right)^{1/2} = \left( \mathbb{E} X_{\infty}^2 \right)^{1/2}
\end{gather*}

\textbf{Proposition} $M^2$ is a Hilbert space and $M^2_c$ is a closed subspace.

\textbf{2.4 Quadratic variation}

\textbf{Definition} For a sequence of processes $(X^n)$ and a process $X$, we say $X^n \to X$ ucp (uniformly on compact sets in probability) if for all $t > 0$ and $\epsilon > 0$, 
\begin{gather*}
\mathbb{P}(\sup_{s \in [0, t]} \abs{X^n_s - X_s} > \epsilon) \to 0
\end{gather*}
as $n \to \infty$.

\newpage

\textbf{Day 4, 31st December 2024, Tuesday}

It is 11.30pm on 31st December 2024, Tuesday. The last day of 2024. The new year is approaching and I feel mostly fear for what the future brings. Fear and stress. The past year seems to have gone by quickly and unproductively. I have accomplished little in the past 4 to 5 years but somehow it feels as if I accomplished the least in the past year. Somehow it feels I will accomplish even less in the coming year, perhaps even degenerate and regress.

I have found it hard to focus in the past few days since coming back to Singapore. Many reasons for my depression and unproductivity have occurred to me. The first reason that came to mind was loneliness. Christmas day was one of the most depressing days in recent memory, almost on par with that worst day of the fall 2024 semester. I spent it alone, aimlessly wandering from the Starbucks at Clarke Quay Central to the tables at Guoco tower at Tanjong Pagar, to the Starbucks at Marina Bay Link Mall. Revisiting all the usual old places. These places used to bring joy to me but now they only bring sadness. I saw so many old lonely people eating alone, sitting alone, aimlessly. It feels as though I am destined to become one of them, or perhaps I am already one of them. Lonely and poor with no purpose in life. 

But it is also perhaps that I have come back to Singaproe too often in the past year. I was only back here in July-August earlier this year, barely 4 months ago. I have accomplished nothing =in the meantime while I was away. It feels as if the city has moved on and left me behind. My peers from Raffles and Cambridge certainly have. I am once again studying at the library. It used to be such a joyful nostalgic thing, learning. But now it seems I am destined, doomed to do this for the rest of my life. Aimlessly, hopelessly studying into oblivion with nothing to show for it. Poor and lonely. Unaccomplished. 

Studying math used to bring solace. It was pure, clean, escapism into a beautiful world where I could be myself. But not anymore. I find myself reading the same things again that I was reading 4 months ago, 1 year ago, 2 years ago. It feels like I am going in circles, or perhaps marching on the spot. Going nowhere. And I do have somewhere I need to be, somewhere I want to be. Somewhere else. There here and now do not spark joy. I long for a better life, a better place. But is that even possible at all now. It all feels so hopelessly impossibly out of reach. I am destined to fail. Forever.                             

Or perhaps it is just the same greed and lack of focus that has always plagued me. I think I have reached my limit. I am being torn apart. My answer to not knowing what I really want to do, all these years, has been to (try to) do everything. It is exhausting. I am exhausted.

It is the new year tomorrow. 1st of January. Just 15 minutes away now. The library won't be open. Sadly? Or perhaps thankfully. I think I will go cycling. I hope the weather will be good. I just want to run away from all of it. But where can I run to? I am stuck. Hopelessly stuck. And alone.

Youth is a wonderful thing. Too bad it is wasted on the young. I wish I could be a secondary school kid at Raffles again. Here I am at 30, turning 31, wishing I was 20 or 15, when one day I shall be 40 wishing I was 30. 

$(X_n)_{n \ge 0}$ is $Markov(\lambda, P)$ if and only if for all $n \ge 0$ and $i_0, \dots, i_n \in I$,
\begin{gather*}
P(X_0 = i_0, \dots, X_n = i_n) = \lambda_{i_0} p_{i_0 i_1} \dots p_{i_{n-1} i_n}
\end{gather*}

Let $\mathcal{A}$ be a $\pi$-system. Then any $d$-system containing $\mathcal{A}$ contains also the $\sigma$-algebra generated by $\mathcal{A}$. 

Let $X$ be an integrable random variable and let $\mathcal{G} \subseteq \mathcal{F}$ be a $\sigma$-algebra. Then there exists a random variable $Y$ such that
\begin{enumerate}[label=(\alph*)]
\item $Y$ is $\mathcal{G}$-measurable
\item $Y$ is integrable and $\mathbb{E}(X1_A) = \mathbb{E}(Y1_A)$ for all $A \in \mathcal{G}$. 
\end{enumerate}

Let $X$ be a continuous local martingale with $X_0 = 0$. If $X$ is also a finite variation process then $X_t = 0$ $\forall t$ a.s.

\textbf{1st January 2025, Wednesday}

Today was better. Went cycling.

\textbf{2nd January 2025, Thursday}

\textbf{ALTAM Chapter 2 Survival Models}

\textbf{2.1 Summary} \\
\textbf{2.2 The future lifetime random variable} \\
\textbf{2.3 The force of mortality} \\
\textbf{2.4 Actuarial notation} \\
\textbf{2.5 Mean and standard deviation of $T_x$} \\
\textbf{2.6 Curtate future lifetime}

\textbf{15 regular exercises, 3 excel based exercises}

Let $(x)$ denote a life aged $x$, where $x \ge 0$. The death of $(x)$ can occur at any age greater than $x$, and we model the future lifetime of $(x)$ by a continuous random variable which we denote by $T_x$. This means that $x + T_x$ represents the age-at-death random variable for $(x)$. We write
\begin{gather*}
F_x(t) = \mathbb{P}(T_x \le t)
\end{gather*}
for the lifetime distribution from age $x$. The survival function
\begin{gather*}
S_x(t) = 1 - F_x(t) = \mathbb{P}(T_x > t)
\end{gather*}
is the probability that $(x)$ survives for at least $t$ years. 

The fundamental assumption: the events $\{T_x \le t\}$ and $\{T_0 \le x + t\}$ are equivalent in the sense that
\begin{gather*}
\mathbb{P}(T_x \le t) = \mathbb{P}(T_0 \le x + t \vert T_0 > x)
\end{gather*}
i.e. given that a newborn individual survives to age $x$, the probability that they die by age $x+t$ is the same as the probability that an individual currently aged $x$ dies within $t$ years. Equivalently,
\begin{gather*}
F_x(t) 
= \mathbb{P}(T_x \le t) 
= \mathbb{P}(T_0 \le x + t \vert T_0 > x) 
= \frac{\mathbb{P}(x < T_0 \le x + t)}{\mathbb{P})T_0 > x)}
= \frac{F_0(x+t) - F_0(x)}{1 - F_0(x)}
\end{gather*}
\begin{gather*}
S_x(t) = 1 - F_x(t) = \frac{1 - F_0(x+t)}{1 - F_0(x)} = \frac{S_0(x+t)}{S_0(x)}
\end{gather*}
\begin{gather*}
S_0(x+t) = S_0(x) S_x(t)
\end{gather*}
i.e. the probability of a newborn surviving to age $x+t$ is the probability that they survive to $x$ and the probability that a person aged $x$ survives another $t$ years. The events of a newborn surviving two disjoint contiguous time periods are independent. More generally,
\begin{gather*}
S_0(x+t+u) = S_0(x+t) S_{x+t}(u) = S_0(x) S_x(t+u) \quad\Rightarrow\quad S_x(t+u) = \frac{S_0(x+t)}{S_0(x)}S_{x+t}(u) = S_x(t) S_{x+t}(u)
\end{gather*}
The events of a life of any age surviving two disjoint contiguous time periods are independent.

We denote the force of mortality at age $x$ by $\mu_x$ and define it as
\begin{align*}
\mu_x 
&= \lim_{dx \to 0^+} \frac{\mathbb{P}(T_x \le x + dx \vert T_0 > x)}{dx} \\
&= \lim_{dx \to 0^+} \frac{\mathbb{P}(T_x \le dx)}{dx} \\
&= \lim_{dx \to 0^+} \frac{\mathbb{P}(T_x \le dx) - \mathbb{P}(T_x \le 0)}{dx} \\
&= \lim_{dx \to 0^+} \frac{F_x(dx) - F_x(0)}{dx} \\
&= - \lim_{dx \to 0^+} \frac{S_x(dx) - S_x(0)}{dx} \\
&= - \lim_{dx \to 0^+} \frac{S_x(dx) - 1}{dx} \\
&= - \frac{1}{S_0(x)} \lim_{dx \to 0^+} \frac{S_0(x+dx) - S_0(x)}{dx} = -\frac{1}{S_0(x)} S_0'(x)
\end{align*}
or equivalently since $S_0(0) = 1$, 
\begin{gather*}
S_0(x) = e^{-\int_0^x \mu_y dy}, \qquad x \ge 0
\end{gather*}
More generally,
\begin{gather*}
S_x(t) = \frac{S_0(x+t)}{S_0(x)} = e^{-\int_x^{x+t} \mu_y dy}
\end{gather*}
In any case, the survival function $S_x(t)$ is kind of exponentially decreasing over time and the force of mortality $\mu_x$ contains all of the information about the rate at which people die off at any age so the fraction of people at age $x$ surviving for at least $t$ years is (one over) the proportion beating the (exponential of) the total force of mortality over those years. 

Two of the most useful mortality laws are Gompertz's law
\begin{gather*}
\mu_x = Bc^x \quad\mbox{ for some constants }\quad B > 0 \quad\mbox{ and }\quad c > 1
\end{gather*}
and Makeham's law, which is a generalisation of Gompertz's law, given by
\begin{gather*}
\mu_x = A + Bc^x \quad\mbox{ for some constants }\quad A, B > 0 \quad\mbox{ and }\quad c > 1
\end{gather*}

One of the earliest mortality laws proposed was De Moivre's law
\begin{gather*}
\mu_x = \frac{1}{\omega - x}, \qquad 0 \le x < \omega
\end{gather*}
This is unrealistic since it implies that $T_x$ is uniformly distributed on the interval $(0, \omega - x)$, which means that he probability that a life currently aged $x$ dies between ages $x+t$ and $x+t+dt$ is the same for all $t > 0$ as long as $x+t > \omega$. 

Another simple mortality law that is very unrealistic for modelling human mortality is the constant force of mortality assumption which states that $\mu_x = \mu$ for all $x \ge 0$. Under this model, $T_x$ has an exponential distribution. 

Although the De Moivre and constant force of mortality models are not useful for overall mortality for humans, they may be used in other contexts, such as modelling the failure time of machine components. We also use these models in a limited sense, to model mortality between integer ages. 

The actuarial notation for survival and mortality probabilities is
\begin{gather*}
_tp_x = S_x(t), \qquad _tq_x = F_x(t), \qquad _u\vert_tq_x = S_x(u) - S_x(u+t)
\end{gather*}
the last being the probability that a life currently aged $x$ survives for $u$ years but then dies within the $t$ years after that. We drop the subscript $t$ if its value is 1 so $p_x$ is the probability that $(x)$ survives to at least age $x+1$ and $q_x$ is the probability that $(x)$ dies before age $x+1$. We call $q_x$ the mortality rate at age $x$ and $_u\vert_tq_x$ the deferred mortality probability. 

The complete expectation of life
\begin{gather*}
\mathring{e}_x = \mathbb{E}(T_x) = \int_0^{\infty} t f_x(t) dt = \int_0^{\infty} t dF_x(t)
= - \int_0^{\infty} t d_tp_x = -\left( t_tp_x \Big\vert_0^{\infty} - \int_0^{\infty} \hspace{1mm}_tp_x dt \right)
= \int_0^{\infty} \hspace{1mm}_tp_x dt
\end{gather*}
is the expected future lifetime of $(x)$.

We are sometimes interested in the future lifetime random variable subject to a cap of $n$ years, which is represented by the random variable $\min(T_x, n)$. The term expectation of life
\begin{gather*}
\mathring{e}_{x:n|}
= \int_0^n tf_x(t) dt + \int_n^{\infty} n f_x(t)dt
= -\int_0^n t d_tp_x + n_np_x
= -\left( t_tp_x \Big\vert_0^n - \int_0^n \hspace{1mm}_tp_x dt \right) + n_np_x
= \int_0^n \hspace{1mm}_tp_x dt
\end{gather*}
is the expected value of $\min(T_x, n)$.

The curtate future lifetime random variable
\begin{gather*}
K_x = \lfloor T_x \rfloor
\end{gather*}
is the integer part of future lifetime. For $k = 0, 1, 2, \dots$ we note that $K_x = k$ iff $(x)$ dies between the ages of $x+k$ and $x+k+1$, so
\begin{gather*}
\mathbb{P}(K_x = k)
= \mathbb{P}(k \le T_x < k+1)
=  \hspace{0mm}_kp_x - \hspace{0mm}_{k+1}p_x
\end{gather*}

The curtate expectation of life is
\begin{gather*}
e_x = \mathbb{E}(K_x) 
= \sum_{k=0}^{\infty} k \mathbb{P}(K_x = k)
= (_1p_x -  \hspace{0mm}_2p_x) + 2( \hspace{0mm}_2p_x -  \hspace{0mm}_3p_x) + 3( \hspace{0mm}_3p_x -  \hspace{0mm}_4p_x) + \dots
= \sum_{k=1}^{\infty}  \hspace{0mm}_kp_x
\end{gather*}

Similarly we have the random variable $\min(K_x, n)$ with expectation
\begin{gather*}
e_{x:n|} = \sum_{k=1}^n  \hspace{0mm}_kp_x
\end{gather*}

From the trapezium approximation for integrals, we obtain the approximation
\begin{gather*}
\mathring{e}_x \approx e_x + \frac{1}{2}
\end{gather*}

\textbf{Exercise 2.1} You are given that
\begin{gather*}
p_x = 0.99, \qquad p_{x+1} = 0.985, \qquad _3p_{x+1} = 0.95, \qquad q_{x+3} = 0.02
\end{gather*}
Calculate $p_{x+3}$, $_2p_x$, $_2p_{x+1}$, $_3p_x$ and $_1\vert_2q_x$

\textbf{Solution 2.1}

\begin{gather*}
x \to x+1 \to x+2 \qquad 1 \to 0.99 \to 0.99 \times 0.985 
\end{gather*}

\begin{gather*}
x+1 \to x+2 \to x+3 \to x+4 \qquad 1 \to ? \to ? \to 0.95
\end{gather*}

\begin{gather*}
x+3 \to x+4 \qquad 1 \to 0.98
\end{gather*}

So immediately have $p_{x+3} = 1 - q_{x+3} = 1 - 0.02 = 0.98$

And quite immediately have $_2p_x = p_x p_{x+1} = 0.99 \times 0.985 = 0.97515$

And quite immediately have $_2p_{x+1} = \hspace{1mm}_3p_{x+1} / p_{x+3} = 0.95 / 0.98 = 0.9693877551$

Then we have $p_{x+2} = \hspace{1mm}_2p_{x+1} / p_{x+1} = 0.9693877551 / 0.985 = 0.98415000518$

And hence $_3p_x = \hspace{1mm}_2p_x p_{x+2} = 0.97515 \times 0.98415000518 = 0.95969387755$

Finally have $_1\vert_2 q_x = p_x (1 - \hspace{1mm}_2p_{x+1}) = 0.99 \times (1 - 0.9693877551) = 0.03030612245$

\textbf{Exercise 2.6} The function
\begin{gather*}
G(x) = \frac{18000 - 110x - x^2}{18000}
\end{gather*}
has been proposed as the survival function $S_0(x)$ for a mortality model.

What is the implied limiting age $\omega$? Verify that the function $G$ satisfies the criteria for a survival function. Calculate $_{20}p_0$. Determine the survival function for a life aged 20. Calculate $_{10}\vert_{10}q_{20}$. Calculate $\mu_{50}$.

\textbf{Solution 2.6}

\begin{gather*}
G(x) = \frac{(200 + x)(90 - x)}{18000}
\end{gather*}
so the limiting age is 90, everyone dies by the time they are 90 years old. $G$ is a survival function since $G(0) = 1$ and $G$ is clearly smooth on $x \in (0, 90)$ and bounded on $x \in [0, 90]$ and hence has finite moments of all orders, and
\begin{gather*}
G'(x) = -\frac{110 + 2x}{18000} < 0
\end{gather*}
for $x \in [0, 90]$ so $G$ is decreasing. The probability that a newborn survives to 20 years old is
\begin{gather*}
_{20}p_0 = G(20) = \frac{220 \times 70}{18000} = 0.85555555555
\end{gather*}
The survival function for a life aged 20 is
\begin{gather*}
_tp_{20} = \frac{_{20+t}p_0}{_{20}p_0} = \frac{G(20+t)}{G(20)} = \frac{(220 + t)(70 - t)}{220 \times 70} = \frac{15400 - 150 t - t^2}{15400}
\end{gather*}
The probability that a life aged 20 survives the next 10 years and then dies within the 10 years after that, is
\begin{gather*}
_{10}\vert_{10}q_{20} = \hspace{1mm}_{10}p_{20} (1 - \hspace{1mm}_{10}p_{30})
= \frac{G(30)}{G(20)} \left(1 - \frac{G(40)}{G(30)}\right)
= \frac{230 \times 60}{220 \times 70} \left(1 - \frac{240 \times 50}{230 \times 60}\right)
= 0.116883117
\end{gather*}
The force of mortality for a life aged 50 is
\begin{gather*}
\mu_{50} = -\frac{G'(50)}{G(50)} = \frac{110 + 2x}{(200+x)(90-x)} = \frac{210}{250 \times 40} = 0.021
\end{gather*}

\textbf{3rd January 2025, Friday, 2.40pm at 5th floor of Tampines Library at Tampines Hub}

\textbf{ALTAM Chapter 3 Life tables and selection}

\textbf{3.1 Summary} \\
\textbf{3.2 Life tables} \\
\textbf{3.3 Fractional age assumptions} \\
\textbf{3.4 National life tables} \\
\textbf{3.5 Survival models for life insurance policyholders} \\
\textbf{3.6 Life insurance underwriting} \\
\textbf{3.7 Select and ultimate survival models} \\
\textbf{3.8 Notation and formulae for select survival models} \\
\textbf{3.9 Select life tables} \\
\textbf{3.10 Some comments on heterogeneity in mortality} \\ 
\textbf{3.11 Mortality improvement modelling} \\
\textbf{3.12 Mortality improvement scales}

March 2025 FAM syllabus excludes sections 4, 10, 11 and 12. 

Given a survival model with survival probabilities $_tp_x$, we construct the life table for the model, from some initial age $x_0$ to a maximum or limiting age $\omega$, using a function $\{l_x\}$, $x_0 \le x \le \omega$, where $l_{x_0}$ is an arbitrary positive number called the radix of the table, and
\begin{gather*}
l_{x_0 + t} = l_{x_0} \hspace{0mm}_tp_{x_0}
\end{gather*}
More generally,
\begin{gather*}
l_{x+t} = l_{x_0} \hspace{0mm}_{x+t-x_0}p_{x_0}
= l_{x_0} \hspace{0mm}_{x-x_0}p_{x_0} \hspace{0mm}_tp_x
= l_x \hspace{0mm}_tp_x \quad\Rightarrow\quad \hspace{0mm}_tp_x = \frac{l_{x+t}}{l_x}
\end{gather*}

In some cases, a life table tabulated at integer ages also shows values of
\begin{gather*}
d_x = l_x - l_{x+1}
\end{gather*}

The uniform distribution of deaths (UDD) assumption is the most common fractional age assumption. It can be formulated in two equivalent ways:

\textbf{UDD1} For integer $x$ and for $0 \le s < 1$, assume that
\begin{gather*}
\hspace{0mm}_sq_x = sq_x
\end{gather*}
That is, for any integer $x$, the mortality probability over $s < 1$ years is $s$ times the one-year mortality probability.

\textbf{UDD2} For a life $(x)$ where $x$ is an integer, with future lifetime random variable $T_x$ and curtate future lifetime random variable $K_x$, define a new random variable $R_x$ to represent the fractional part of the future lifetime of $(x)$ lived in the year of death, so that $T_x = K_x + R_x$. We assume
\begin{gather*}
R_x \sim U(0, 1) \quad\mbox{ independent of }\quad K_x
\end{gather*}

An immediate consequence is that 
\begin{gather*}
l_{x+s} = l_x - sd_x, \qquad 0 \le s < 1
\end{gather*}
Thus UDD implies that $l_{x+s}$ is a linearly decreasing function of $s$ between integer ages. 

\color{red}
Furthermore,
\begin{gather*}
\mu_{x+s} = 
\end{gather*}
\color{black}

A second fractional age assumption is that the force of mortality is constant between integer ages. Thus for integer $x$ and $0 \le s < 1$ we assume that $\mu_{x+s}$ does not depend on $s$, and we denote it $\mu_x^*$. We can obtain the value of $\mu_x^*$ from the life table by using the fact that
\begin{gather*}
p_x = e^{-\int_0^1 \mu_{x+t} dt} = e^{-\mu_x^*} \quad\Rightarrow\quad \mu_x^* = -\ln p_x
\end{gather*}
Furthermore, 
\begin{gather*}
\hspace{0mm}_tp_x = e^{-\int_0^t \mu_{x+s} ds} = e^{-t\mu_x^*} = (p_x)^t
\end{gather*}
More generally for $r, t > 0$ and $r + t < 1$, 
\begin{gather*}
\hspace{0mm}_tp_{x+r} = e^{-\int_0^t \mu_{x+r+s} ds} = e^{-t\mu_x^*} = (p_x)^t
\end{gather*}

Life tables based on the mortality experience of the whole population are regularly produced for countries around the world. Separate life tables are usually produced for males and for females and possibly for some other groups of individuals e.g. by race or socio-economic group.

Suppose we have to choose a survival model appropriate for a man who is currently aged 50 and living in the UK, and who has just purchased a 10-year term insurance policy. We could use a national life table. However, in the UK, as in other countries with well-developed life insurance markets, the mortality experience of people who purchase life insurance policies tends to be different from the population as a whole. The mortality of different types of life insurance policyholders is investigated separately, and life tables appropriate for these groups are published. 

The mortality rates in the CMI tables are based on individuals accepted for insurance at normal premium rates, that is, individuals who have passed the required health checks. This means that a man aged 50 who has just purchased a term insurance at the normal premium rate is known to be in good health, assuming the health checks are effective, and so is likely to be much healthier and hence have a lower mortality rate than a man of age 50 picked randomly from the population. When this man reaches age 56, we can no longer be certain he is in good health. All we know is that he was in good health six years ago. Hence his mortality rate at age 56 is higher than that of a man of the same age who has just passed the health checks and been permitted to buy a term insurance policy at normal rates. The effect of passing the health checks at issue eventually wears off, so that at age 62, the force of mortality does not depend on whether the policy was purchased at age 50, 52, 54 or 56. 

A feature of the survival models studied in Chapter 2 is that probabilities of future survival depend only on the individual's current age. For example, for a given survival model and a given term $t$, the probability $_tp_x$ that an individual currently aged $x$ will survive to age $x+t$ depends only on the current age $x$. Such survival models are called aggregate survival models, because lives are all aggregated together. 

The difference between an aggregate survival model and the survival model for term insurance policyholders discussed in Section 3.6 is that in the latter case, probabilities of future survival depend not only on current age but also on how long ago the policy was purchased, which was when the policyholder joined the group of insured lives. 

The mortality of a group of individuals is described by a ``select and ultimate survival model", usually shortened to ``select survival model", if:
\begin{enumerate}
\item Future survival probabilities for an individual in the group depend on the individual's current age and on the age at which the individual joined the group.
\item There is a positive number $d$, generally an integer, such that if an individual joined the group more than $d$ years ago, future survival probabilities depend only on current age. The initial selection effect is assumed to have worn off after $d$ years. 
\end{enumerate}
An individual who enters the group at age $x$ is said to be ``select" at age $x$. The period $d$ after which the age at selection has no effect on future survival probabilities is called the ``select period" for the model. The mortality that applies to lives after the select period is complete is called ``ultimate mortality", so that the complete model comprises a select period followed by the ultimate period. 

A select survival model represents an extension of the ultimate survival model studied in Chapter 2. This requires the notation
\begin{gather*}
_tp_{[x]+s} = \mathbb{P}(\mbox{a life $(x+s)$ select at aged $x$ survives to age $x+s+t$})
\end{gather*}
\begin{gather*}
_tq_{[x]+s} = \mathbb{P}(\mbox{a life $(x+s)$ select at aged $x$ does not survive to age $x+s+t$})
\end{gather*}
\begin{gather*}
\mu_{[x]+s} = \lim_{h\to0^+} \frac{1 - \hspace{0mm}_hp_{[x]+s}}{h} = \mbox{force of mortality at age $x+s$ for a life select at age $x$}
\end{gather*}
From these definitions we can obtain
\begin{gather*}
_tp_{[x]+s} = e^{-\int_0^t \mu_{[x]+s+u} du}
\end{gather*}

For an ultimate survival model the life table $\{l_x\}$ is useful since it can be used to calculate probabilities of survival. We can construct a select life table in a similar way but we need the table to reflect duration since selection, as well as age, during the select period. Select life tables are usually presented at integer ages only, just as for the aggregated life tables. 

\textbf{14 regular exercises, 0 excel based exercises}

\textbf{Exercise 3.2} You are given the following life table extract:
\begin{gather*}
l_{52} = 89948, \qquad l_{53} = 89089, \qquad l_{54} = 88176, \qquad l_{55} = 87208, \\ 
l_{56} = 86181, \qquad l_{57} = 85093, \qquad l_{58} = 83940, \qquad l_{59} = 82719 
\end{gather*}
Calculate each of the following probabilities assuming (i) uniform distribution of deaths between integer ages, and (ii) a constant force of mortality between integer ages:
\begin{gather*}
\hspace{0mm}_{0.2}q_{52.4}, \qquad \hspace{0mm}_{5.7}p_{52.4}, \qquad \hspace{0mm}_{3.2}\vert_{2.5}q_{52.4}
\end{gather*}

\textbf{Solution 3.2} Assuming uniform distribution of deaths between integer ages,
\begin{align*}
l_{52.4} &= 0.6 \times 89948 + 0.4 \times 89089 = 89604.4, \\ 
l_{52.6} &= 0.4 \times 89948 + 0.6 \times 89089 = 89432.6, \\
l_{55.6} &= 0.4 \times 87208 + 0.6 \times 86181 = 86591.8, \\
l_{58.1} &= 0.9 \times 83940 + 0.1 \times 82719 = 83817.9
\end{align*}
so
\begin{align*}
\hspace{0mm}_{0.2}q_{52.4} &= 1 - \frac{l_{52.6}}{l_{52.4}} = 0.001917317, \\ 
\hspace{0mm}_{5.7}p_{52.4} &= \frac{l_{58.1}}{l_{52.4}} = 0.935421698, \\ 
\hspace{0mm}_{3.2}\vert_{2.5}q_{52.4} &= \frac{l_{55.6}}{l_{52.4}} \left(1 - \frac{l_{58.1}}{l_{55.6}}\right) = 0.030957185
\end{align*}

Assuming a constant force of mortality between integer ages, 
\begin{align*}
l_{52.4} &= 89948^{0.6} 89089^{0.4} = 89603.41054, \\
l_{52.6} &= 89948^{0.4} 89089^{0.6} = 89431.61118, \\
l_{55.6} &= 87208^{0.4} 86181^{0.6} = 86590.34064, \\
l_{58.1} &= 83940^{0.9} 82719^{0.1} = 83817.09332
\end{align*}
so
\begin{align*}
\hspace{0mm}_{0.2}q_{52.4} &= 1 - \frac{l_{52.6}}{l_{52.4}} = 0.001917331, \\ 
\hspace{0mm}_{5.7}p_{52.4} &= \frac{l_{58.1}}{l_{52.4}} = , 0.935423025 \\ 
\hspace{0mm}_{3.2}\vert_{2.5}q_{52.4} &= \frac{l_{55.6}}{l_{52.4}} \left(1 - \frac{l_{58.1}}{l_{55.6}}\right) = 0.030950243
\end{align*}

\textbf{Exercise 3.8} A group of 100,000 independent lives each aged 65 purchases one-year term insurance. At the start, 20\% of the group are preferred lives with $q_{65} = 0.002$ and 80\% of the group are normal lives with $q_{65} = 0.009$. 
\begin{enumerate}[label = (\alph*)]
\item Calculate the standard deviation of the number of survivors at the end of the year.
\item Calculate the proportion of preferred lives expected in the survivor group.
\item Using a normal approximation without continuity correction, calculate the 90th percentile of the number of survivors at the end of the year.
\end{enumerate}

\textbf{Solution 3.8}
\begin{gather*}
\mathbb{P}(\mbox{preferred}) = 0.2, \qquad \mathbb{P}(\mbox{normal}) = 0.8, \qquad
\mathbb{P}(\mbox{survive}|\mbox{preferred}) = 0.998, \qquad
\mathbb{P}(\mbox{survive}|\mbox{normal}) = 0.991
\end{gather*}
\begin{gather*}
\mathbb{P}(\mbox{survive}) = 0.9924, \qquad \mbox{stdev} = \sqrt{100,000 \times 0.9924(1 - 0.9924)} = 27.463
\end{gather*}
\color{red}
\textbf{Actually maybe we should calculate the standard deviation of each type of life separately and add up within each population as in}
\begin{gather*}
\sqrt{20,000 \times 0.998 \times 0.002 + 80,000 \times 0.991 \times 0.009} = 27.44886154
\end{gather*}
\color{black}
Expected number of surviving preferred lives $= 0.998 \times 20,000 = 19960$ \\
Expected number of surviving normal lives $= 0.991 \times 80,000 = 79280$ \\
Proportion of preferred lives expected in survivor group $= 19960 / (19960 + 79280) = 0.201128577$
\begin{gather*}
\mbox{mean number of survivors} = 99240, \qquad \mbox{stdev number of survivors} = 27.463
\end{gather*}
Standard normal 90th percentile is 1.28 so our 90th percentile $x$ is given by
\begin{gather*}
\frac{x - 99240}{27.463} = 1.28 \quad\Rightarrow\quad x = 99275.15264
\end{gather*}

\textbf{4th January 2025, Saturday 2.58pm at 4th floor of Bishan Library}

\textbf{ALTAM Chapter 4 Insurance benefits}

\textbf{4.1 Summary} \\
\textbf{4.2 Introduction} \\
\textbf{4.3 Assumption} \\
\textbf{4.4 Valuation of insurance benefits} \\
\textbf{4.5 Relating $\overline{A}_x$, $A_x$ and $A^{(m)}_x$} \\
\textbf{4.6 Variable insurance benefits} \\
\textbf{4.7 Functions for select lives}

In this chapter we develop formulae for the valuation of traditional insurance benefits: whole life, term and endowment insurance. 

In the previous two chapters we looked at models for future lifetime. The main reason that we need these models is to apply them to the valuation of payments which are dependent on the death or survival of a policyholder or pension plan member. 

To perform calculations in this chapter, we require assumptions about mortality and interest. For mortality, we assume the standard ultimate survival model given by Makeham's law
\begin{gather*}
\mu_x = A + Bc^x, \qquad A = 0.00022, \qquad B = 2.7 \times 10^{-6}, \qquad c = 1.124
\end{gather*}
We also assume that interest rates are constant. 

Given an effective annual rate of interest $i > 0$ we use $\nu = 1/(1+i)$ so that the present value of a payment of $S$ which is to be paid in $t$ years' time is $S\nu^t$. The force of interest per year, also known as the continuously compounded interest rate, is
\begin{gather*}
\delta = \ln(1+i), \qquad 1+i = e^{\delta}, \qquad \nu = e^{-\delta}
\end{gather*}
The nominal rate of interest compounded $p$ times per year is denoted $i^{(p)}$ and is defined by
\begin{gather*}
1+ i = (1+i^{(p)}/p)^p
\end{gather*}
The effective rate of discount per year is
\begin{gather*}
d = 1 - \nu = i\nu = 1 - e^{-\delta}
\end{gather*}
and the nominal rate of discount compounded $p$ times per year is $d^{(p)}$ defined by
\begin{gather*}
(1-d^{(p)}/p)^p = \nu
\end{gather*}

The ``expected present value" or ``actuarial present value" of a benefit of \$1 payable following the death of a life currently aged $x$ is
\begin{gather*}
\overline{A}_x = \mathbb{E}(e^{-\delta T_x}) = \int_0^{\infty} e^{-\delta t} \hspace{0mm}_tp_x \mu_{x+t} dt
\end{gather*}
The second moment of the present value of the death benefit is
\begin{gather*}
^2\overline{A}_x = \int_0^{\infty} e^{-2\delta t} \hspace{0mm}_tp_x \mu_{x+t} dt
\end{gather*}
The variance of the present value of a unit benefit payable immediately on death is
\begin{gather*}
^2\overline{A}_x - (\overline{A}_x)^2
\end{gather*}

Suppose now that the benefit if \$1 is payable at the end of the year of death of $(x)$ rather than immediately on death. The EPV of the benefit is
\begin{gather*}
A_x = \mathbb{E}(\nu^{K_x + 1}) = \nu q_x + \nu^2 \hspace{0mm}_1\vert q_x + \nu^3 \hspace{0mm} _2\vert q_x + \dots
\end{gather*}
The second moment is
\begin{gather*}
^2A_x = \mathbb{E}(\nu^{K_x + 1}) = \nu^2 q_x + \nu^4 \hspace{0mm}_1\vert q_x + \nu^6 \hspace{0mm} _2\vert q_x + \dots
\end{gather*}
The variance is
\begin{gather*}
^2A_x - (A_x)^2
\end{gather*}

Using an annual life table, we can calculate the values of $A_x$ using backward recursion:
\begin{gather*}
A_x = \nu q_x + \nu p_x A_{x+1}
\end{gather*}

Under a term insurance policy, the death benefit is payable only if the policyholder dies within a fixed term of say $n$ years. In the continuous case, the benefit is payable immediately on death. The EPV of this benefit is
\begin{gather*}
\overline{A}_{x:n|}^1 = \int_0^n e^{-\delta t} \hspace{0mm}_tpx \mu_{x+t} dt
\end{gather*}
and the expected value of the square of the present value is
\begin{gather*}
\hspace{0mm}^2\overline{A}_{x:n|}^1 = \int_0^n e^{-2\delta t} \hspace{0mm}_tpx \mu_{x+t} dt
\end{gather*}

If the death benefit of 1 is payable at the end of the year of death, provided this occurs within $n$ years, the EPV is
\begin{gather*}
A_{x:n|}^1 = \sum_{k=0}^{n-1} \nu^{k+1} \hspace{0mm}_k\vert q_x
\end{gather*}

Pure endowment benefits are conditional on the survival of the policyholder at the policy maturity date. The EPV of a pure endowment of \$1 issued to a life aged $x$ with a term of $n$ years, is
\begin{gather*}
\hspace{0mm}_nE_x = \nu^n \hspace{0mm}_np_x
\end{gather*}

An endowment insurance provides a combination of a term insurance and a pure endowment. The sum insured is payable on the death of $(x)$ if $(x)$ dies within a fixed term, say $n$ years, but if $(x)$ survives $n$ years, the sum insured is payable at the end of the $n$th year. 

In the case where the death benefit of amount 1 is payable immediately on death, the EPV is
\begin{gather*}
\overline{A}_{x:n|} = \overline{A}_{x:n|}^1 + \hspace{0mm}_nE_x
\end{gather*}

In the situation where the death benefit is payable at the end of the year of death, the EPV is
\begin{gather*}
A_{x:n|} = A_{x:n|}^1 + \hspace{0mm}_nE_x
\end{gather*}

Deferred insurance refers to insurance which does not begin to offer death benefit cover until the end of a deferred period. 

The force of mortality $\mu_x$ is defined by
\begin{gather*}
S_0(x) = e^{-\int_0^x \mu_s ds} \quad\Leftrightarrow\quad S_0'(x) = -\mu_x S_0(x) \quad\Leftrightarrow\quad
\mu_x = -\frac{1}{S_0(x)} \lim_{h \to 0^+} \frac{S_0(x+h) - S_0(x)}{h}
\end{gather*}
so more generally we have
\begin{gather*}
S_x(t) = \frac{S_0(x+t)}{S_0(x)} = e^{-\int_x^{x+t} \mu_s ds}
\end{gather*}
In any case,
\begin{gather*}
f_x(t) = F_x'(t) = - S_x'(t) = \mu_{x+t} S_x(t) = \hspace{0mm}_tp_x \mu_{x+t}, \qquad x, t \in \mathbb{R}_{\ge 0}
\end{gather*}

Under the UDD assumption, for integer $x$ and real $t \in (0, 1)$, have linear interpolation
\begin{gather*}
\hspace{0mm}_tp_x = (1-t) + tp_x = e^{-\int_0^t \mu_{x+s} ds} \quad\Rightarrow\quad
p_x - 1 = (-\mu_{x+t}) \hspace{0mm}_tp_x
\end{gather*}
by differentiating both sides wrt $t$. Since $q_x = 1 - p_x$, it follows that
\begin{gather*}
\hspace{0mm}_tp_x \mu_{x+t} = q_x, \qquad x \in \mathbb{Z}_{\ge 0}, \qquad t \in (0, 1)
\end{gather*}
so
\begin{gather*}
f_x(t) = q_x, \qquad x \in \mathbb{Z}_{\ge 0}, \qquad t \in (0, 1)
\end{gather*}
depends only on the integer $x$ and not on the fractional part $t$. ``Uniform Distribution of Deaths" is a good name indeed. For general $s = k + t \in \mathbb{R}_{\ge t0}$ comprising of integer part $k \in \mathbb{Z}_{\ge 0}$ and fractional part $t \in (0, 1)$, have
\begin{gather*}
f_x(s) = f_x(k+t) = \hspace{0mm}_{k+t}p_x \mu_{x+k+t} = \hspace{0mm}_kp_x \hspace{1mm}_tp_{x+k} \hspace{1mm} \mu_{x+k+t}
= \hspace{0mm}_kp_x q_{x+k} = \mathbb{P}(k \le T_x < k)
\end{gather*}

Hence
\begin{align*}
\overline{A}_x
&= \int_0^{\infty} e^{-\delta s} f_x(s) ds \\
&= \sum_{k = 0}^{\infty} \left( \int_k^{k+1} e^{-\delta s} f_x(s) ds \right) \\
&= \sum_{k = 0}^{\infty} \left( \mathbb{P}(k \le T_x < k) \hspace{1mm} \int_k^{k+1} e^{-\delta s} ds \right) \\
&= \sum_{k = 0}^{\infty} \left( \mathbb{P}(k \le T_x < k) \hspace{1mm} \frac{e^{-\delta k} - e^{-\delta(k+1)}}{\delta} \right) \\
&= \frac{e^{\delta} - 1}{\delta} \sum_{k = 0}^{\infty} e^{-\delta (k+1)} \mathbb{P}(k \le T_x < k)
= \frac{e^{\delta} - 1}{\delta} A_x = \frac{i}{\delta} A_x
\end{align*}

Similarly,
\begin{gather*}
\overline{A}_x = \frac{i^{(m)}}{\delta} A^{(m)}_x \quad\Rightarrow\quad A^{(m)}_x = \frac{i}{i^{(m)}}A_x
\end{gather*}

It took me a while to sort out and derive the above results. Probably just memorise them for the exam. 

We stress that these approximations apply only to death benefits. An endowment insurance combines the death and survival benefits, so we need to split off the death benefit before applying the above formulas. So under the UDD assumption,
\begin{gather*}
\overline{A}_{x:n|} \approx \frac{i}{\delta} A_{x:n|}^1 + \hspace{0mm}_nE_x
\end{gather*}

\color{red}
\textbf{4.5.2 Using the claims acceleration approach}
\color{black}

\textbf{32 regular exercises, 2 excel based exercises}

\textbf{Exercise 4.1} You are given the following table of values for $l_x$ and $A_x$, assuming an effective interest rate of $6\%$ per year. Calculate
\begin{gather*}
\hspace{0mm}_5E_{35}, \qquad A_{35:5|}^1, \qquad \hspace{0mm}_5\vert A_{35}, \qquad \overline{A}_{35:5|}
\end{gather*}
assuming UDD between integer ages where necessary. 

\textbf{Solution 4.1}
\begin{gather*}
\hspace{0mm}_5E_{35} = \frac{1}{1.06^5} \frac{l_{40}}{l_{35}} = 0.735972183
\end{gather*}
\begin{align*}
\mathbb{P}(35 < T_x < 36) &= 1 - l_{36}/l_{35} = 0.0026285 \\
\mathbb{P}(36 < T_x < 37) &= (l_{36}/l_{35})(1- l_{37}/l_{36}) = l_{36}/l_{35} - l_{37}/l_{35} = 0.0028124 \\
\mathbb{P}(37 < T_x < 38) &=  l_{37}/l_{35} - l_{38}/l_{35} = 0.0030119 \\
\mathbb{P}(38 < T_x < 39) &=  l_{38}/l_{35} - l_{39}/l_{35} = 0.0032281 \\
\mathbb{P}(39 < T_x < 40) &=  l_{39}/l_{35} - l_{40}/l_{35} = 0.0034623
\end{align*}
\begin{align*}
A_{35:5|}^1 
&= \frac{\mathbb{P}(35 < T_x < 36)}{1.06} 
+ \frac{\mathbb{P}(36 < T_x < 37)}{1.06^2} 
+ \frac{\mathbb{P}(37 < T_x < 38)}{1.06^3} 
+ \frac{\mathbb{P}(38 < T_x < 39)}{1.06^4} 
+ \frac{\mathbb{P}(39 < T_x < 40)}{1.06^5} \\
&= 0.012655782
\end{align*}
\begin{gather*}
\hspace{0mm}_5\vert A_{35} = \hspace{0mm}_5p_{35} \frac{A_{40}}{1.06^5} = 0.9848569 \times 0.140852188 = 0.138719235
\end{gather*}
\begin{gather*}
\overline{A}_{35:5|}
= \overline{A}_{35:5|}^1 + \hspace{0mm}_5E_{35} 
\approx \frac{i}{\delta} A_{35:5|}^1 + \hspace{0mm}_5E_{35} 
= \frac{0.06}{\ln 1.06} 0.012655782 + 0.735972183
= 0.749003951
\end{gather*}

\textbf{Exercise 4.18} You are given the following excerpt from a select life table. Assuming an interest rate of $6\%$ per year, calculate
\begin{enumerate}[label=(\alph*)]
\item $A_{[40]+1:4|}$
\item the standard deviation of the present value of a four-year term insurance, deferred one year, issued to a newly selected life aged 40, with sum insured \$100,000 payable at the end of the year of death
\item \color{red}the probability that the present value of the benefit described in part (b) is less than or equal to \$85,000 \color{black}
\end{enumerate}

\textbf{Solution 4.18}
\begin{align*}
A_{[40]+1:4|}
&= \left( 1 - \frac{l_{[40]+2}}{l_{[40]+1}} \right) \frac{1}{1.06}
+ \left( \frac{l_{[40]+2}}{l_{[40]+1}} - \frac{l_{[40]+3}}{l_{[40]+1}} \right) \frac{1}{1.06^2} \\
&+ \left( \frac{l_{[40]+3}}{l_{[40]+1}}  - \frac{l_{44}}{l_{[40]+1}} \right) \frac{1}{1.06^3}
+ \left( \frac{l_{44}}{l_{[40]+1}} - \frac{l_{45}}{l_{[40]+1}} \right) \frac{1}{1.06^4} 
+ \frac{l_{45}}{l_{[40]+1}} \frac{1}{1.06^4} \\
&= 0.792669005
\end{align*}
\begin{gather*}
\mathbb{E}(X) = \frac{l_{[40]+1}}{l_{[40]}} \frac{1}{1.06} \left(A_{[40]+1:4|} - \frac{l_{45}}{l_{[40]+1}} \frac{1}{1.06^4} \right)
= 0.007013483
\end{gather*}
\begin{align*}
\mathbb{E}(X^2)
&= \left( \frac{l_{[40]+1}}{l_{[40]}} - \frac{l_{[40]+2}}{l_{[40]}} \right) \frac{1}{1.06^4}
+ \left( \frac{l_{[40]+2}}{l_{[40]}} - \frac{l_{[40]+3}}{l_{[40]}} \right) \frac{1}{1.06^6} \\
&+ \left( \frac{l_{[40]+3}}{l_{[40]}}  - \frac{l_{44}}{l_{[40]}} \right) \frac{1}{1.06^8}
+ \left( \frac{l_{44}}{l_{[40]}} - \frac{l_{45}}{l_{[40]}} \right) \frac{1}{1.06^{10}}  = 0.005703787
\end{align*}
\begin{gather*}
100,000 \times \mbox{stdev}(X) 
= 100,000 \times \sqrt{\mathbb{E}(X^2) - \mathbb{E}(X)^2} 
= 7519.706149
\end{gather*}
\color{red}
Approximating the present value of benefit random variable $X$ as normal with the above expectation and variance, 
\begin{align*}
\mathbb{P}(X < 85,000)
&= \mathbb{P} \left( \frac{X - 701.3483}{7519.706149} < \frac{85,000 - 701.3483}{7519.706149} \right) \\
&= \mathbb{P} \left( \frac{X - 701.3483}{7519.706149} < 11.21\right)
\end{align*}
\color{black}

\textbf{5th January 2025, Sunday 5.08pm at Harbourfront Library at Vivocity}

\textbf{ALTAM Chapter 5 Annuities}

\textbf{5.1 Summary} \\
\textbf{5.2 Introduction} \\
\textbf{5.3 Review of annuities-certain} \\
\textbf{5.4 Annual life annuities}

In this chapter we derive expressions for the valuation and analysis of life contingent annuities.

A life annuity is a series of payments which are contingent on a specified individual's survival to each potential payment date. The payments are normally made at regular intervals and the most common situation is that the payments are of the same amount. 

Recall that an annuity-due is an annuity with each payment made at the start of a time interval (e.g. at the beginning of each month), and an immediate annuity is one where each payment is made at the end of a time interval (e.g. at the end of each month). 

For integer $N$ and effective interest rate $i > 0$ per annum, the present value of an annuity-due of 1 per year is
\begin{gather*}
\ddot{a}_{n|} = 1 + \nu + \nu^2 + \dots + \nu^{n-1} = \frac{1 - \nu^n}{d}
\end{gather*}

The present value of an immediate annuity of 1 per year is
\begin{gather*}
a_{n|} = \nu + \nu^2 + \nu^3 + \dots + \nu^n = \frac{1 - \nu^n}{i}
\end{gather*}

The annual life annuity is paid once each year, conditional on the survival of a life to the payment date. If the annuity is to be paid throughout the annuitant's life, it is called a whole life annuity. If there is to be a specified maximum term, it is called a term annuity. 

Consider first a whole life annuity-due with annual payments of 1 per year, which depend on the survival of a life currently aged $x$. This means that the random variable representing the present value of the whole life annuity-due for $(x)$ is
\begin{gather*}
\ddot{a}_{K_x + 1 |} = \frac{1 - \nu^{K_x + 1}}{d}
\end{gather*}
which has expected value
\begin{gather*}
\ddot{a}_x = \mathbb{E}\left( \frac{1 - \nu^{K_x+1}}{d} \right) = \frac{1 - \mathbb{E}(\nu^{K_x+1})}{d} = \frac{1 - A_x}{d}
\end{gather*}
and variance
\begin{gather*}
\Var\left(\frac{1 - \nu^{K_x+1}}{d}\right) = \frac{\Var(\nu^{K_x+1})}{d^2} = \frac{\hspace{0mm}^2A_x - A_x^2}{d^2}
\end{gather*}

Now suppose we wish to value a term annuity-due of 1 per year. The present value of this annuity is
\begin{gather*}
\ddot{a}_{\min(K_x+1, n)|} = \frac{1 - \nu^{\min(K_x+1, n)}}{d}
\end{gather*}
which has expected value
\begin{gather*}
\ddot{a}_{x:n|} = \frac{1 - \mathbb{E}(\nu^{\min(K_x+1, n)})}{d} = \frac{1 - A_{x:n|}}{d}
\end{gather*}

\textbf{23 regular exercises, 2 excel based exercises}

\textbf{ALTAM Chapter 6 Premium Calculation}

\textbf{6.1 Summary}

\textbf{$n$ regular exercises, $m$ excel based exercises}

\textbf{ALTAM Chapter 7 Policy Values}

\textbf{7.1 Summary}

\textbf{$n$ regular exercises, $m$ excel based exercises}

\textbf{Part 2 Probability \& Measure}

Let $E$ be a set. 

A set $\mathcal{E}$ of subsets of $E$ is called a $\pi$-system if it contains the empty set and the intersection of any two sets in that set. 

A set of subsets of $E$ is called a $d$-system if 

A set of subsets of $E$ is called a $\sigma$-algebra if it contains $\emptyset$ and is closed under complements and countable unions.

Clearly a set of subsets of $E$ is a $\sigma$-algebra iff it is a $\pi$-system and a $d$-system.

Clearly the intersection of a collection of $\sigma$-algebras is a $\sigma$-algebra, so for any set of subsets of $E$ we can define the $\sigma$-algebra generated by that set of subsets to be the intersection of all $\sigma$-algebras of $E$ containing that set of subsets. 

\textbf{Proposition} If $d$-system contains a $\pi$-system then it also contains the $\sigma$-algebra generated by the $\pi$-system. 

Let $E$ be a set.

A set of subsets of $E$ is called a ring if it contains $\emptyset$ and it contains $A\backslash B$ whenever it contains $A$ and $B$.

A set of subsets of $E$ is called an algebra if it contains $\emptyset$ and is closed under complements and finite unions.

Let $\Omega$ be a set and $\mathcal{F}$ a $\sigma$-algebra of subsets. We call the pair $(\Omega, \mathcal{F})$ a measurable space.

A measure on $(\Omega, \mathcal{F})$ is a function $\mu : \mathcal{F} \to [0, \infty]$ such that $\mu(\emptyset) = 0$ and such that whenever $A_1, A_2, \dots$ is a sequence of disjoint sets in $\mathcal{F}$, have
\begin{gather*}
\mu\left(\bigcup_{i=1}^n A_i\right) = \sum_{i=1}^n \mu(A_i)
\end{gather*}
This last property is called countable additivity.

A set function is said to be countably subadditive if for any sequence of sets $A_1, A_2, \dots$ not necessarily disjoint, have
\begin{gather*}
\mu\left(\bigcup_{i=1}^n A_i\right) \le\sum_{i=1}^n \mu(A_i)
\end{gather*}

Clearly, a measure is increasing:
\begin{gather*}
A \subseteq B \quad\Rightarrow\quad \mu(A) \le \mu(B)
\end{gather*}
and countably subadditive. 

\textbf{Proposition} Let $\Omega$ be a set, and $\mathcal{A}$ a ring of subsets of $\Omega$. Then any set function on $\mathcal{A}$ extends to a measure on the $\sigma$-algebra generated by $\mathcal{A}$.

\textbf{ALTAM Syllabus}

\textbf{ALTAM Chapter 3 Life Tables and Selection} (sections 4 and 10) \\
\textbf{ALTAM Chapter 7 Policy Values} (sections 2.4 and 4)

\textbf{ALTAM Chapter 8 Multiple State Models} \\
\textbf{ALTAM Chapter 9 Multiple Decrement Models} \\
\textbf{ALTAM Chapter 10 Joint Life and Last Survivor Benefits} \\
\textbf{ALTAM Chapter 11 Pension Mathematics} (except section 12) \\
\textbf{ALTAM Chapter 13 Emerging Costs for Traditional Life Insurance} (except section 8) \\
\textbf{ALTAM Chapter 14 Universal Life Insurance} \\
\textbf{ALTAM Chapter 15 Emerging Costs for Equity-Linked Insurance} (sections 1-3) \\
\textbf{ALTAM Chapter 17 Embedded Options} \\
\textbf{ALTAM Chapter 18 Estimating Survival Models} (sections 5 and 6 note sections 1-3 are background only)

\textbf{ALTAM Chapter 8 Policy Values}

\textbf{8.1 Summary} \\
\textbf{8.2 Examples of multiple state models} \\
\textbf{8.3 Assumptions and notation} \\
\textbf{8.4 Formulae for probabilities} \\
\textbf{8.5 Numerical evaluation of probabilities} \\
\textbf{8.6 State-dependent insurance and annuity functions} \\
\textbf{8.7 Premiums}

Suppose we have a life aged $x \ge 0$ at time $t = 0$. For each $t \ge 0$ we define a random variable $Y(t)$ which takes one of the two values 0 or 1. The event $\{Y(t) = 0\}$ means that our individual is alive at age $x+t$ and the event $\{Y(t) = 1\}$ means that our individual died before age $x+t$. The set of random variables $\{Y(t)\}_{t \ge 0}$ is an example of a continuous time stochastic process. The future lifetime random variable is then
\begin{gather*}
T_x = \max\{t : Y(t) = 0\}
\end{gather*}

Suppose we are interested in a term insurance policy under which the death benefit is \$100,000 if death is due to an accident during the policy term and \$50,000 if it is due to any other cause. An appropriate model has 3 states, and we can define a continuous time stochastic process $\{Y(t)\}_{t \ge 0}$ where each random variable $Y(t)$ takes one of the values 0, 1 or 2. Hence for example the event $\{Y(t) = 1\}$ indicates that the individual who was aged $x$ at time $t = 0$ has died from an accident before age $x+t$. The difference is that now we distinguish between deaths due to accident and deaths due to other causes. Notice that it is the benefits provided by the insurance policy that determine the design of the model. Because the benefit is contingent on the cause of death, the model must specify the cause of death appropriately. 

Figure 8.3 shows a model appropriate for a policy which provides some or all of the following payments:
\begin{enumerate}[label=(\roman*)]
\item an annuity while permanently disabled
\item a lump sum on becoming permanently disabled
\item a lump sum on death
\item premiums payable while healthy
\end{enumerate}
An important feature of this model is that disability is permanent - there is no arrow from State 1 back to State 0. 

Figure 8.4 shows the sickness-death model which can be used for a policy which provides an annuity during periods of sickness, with premiums payable while the person is healthy. It could also be used for valuing lump sum payments contingent on becoming sick or dying. The model represented by Figure 8.4 differs from that in Figure 8.3 in only one respect: it is possible to transfer from State 1 to State 0, that is, to recover from an illness. This model illustrates an important general feature of multiple state models - the possibility of entering one or more states many times. 

In this section we consider a general multiple state model. We have a finite set of $m+1$ states labelled $0, 1, \dots, m$ with instantaneous transitions being possible between selected pairs of states. These states represent different conditions for an individual. For each $t \ge 0$, the random variable $Y(t)$ takes one of the values $0, 1, \dots, m$ and we interpret the event $Y(t) = i$ to mean that the individual is in State $i$ at age $x+t$. The set of random variables $\{Y(t)\}_{t \ge 0}$ is then a continuous time stochastic process. 

\textbf{Assumption 8.1} We assume that for any states $i$ and $j$ and any times $t$ and $t+s$ where $s \ge 0$, the conditional probability
\begin{gather*}
\mathbb{P}(Y(t+s) = j \vert Y(t) = i)
\end{gather*}
is well defined, in the sense that its value does not depend on any information about the process before time $t$.

Intuitively, this means that the probabilities of future events for the process are completely determined by knowing the current state of the process. This is known as the Markov property. 

\textbf{Assumption 8.2} We assume that for any positive interval of time $h$,
\begin{gather*}
\mathbb{P}(\mbox{2 or more transitions within a time period of length $h$}) = o(h)
\end{gather*}

Recall that a function $g(h)$ is said to be $o(h)$ if
\begin{gather*}
\lim_{h \to 0} \frac{g(h)}{h} = 0
\end{gather*}
Intuitively a function is $o(h)$ if it converges to zero faster than $h$ does when $h$ tends to zero. 

For states $i$ and $j$ in a multiple state model and for $x, t \ge 0$, we define
\begin{gather*}
\hspace{0mm}_tp_x^{ij} = \mathbb{P}(Y(x+t) = j \vert Y(x) = i), \qquad
\hspace{0mm}_tp_x^{\overline{ii}} = \mathbb{P}(Y(x+s) = i \mbox{ for all $s \in [0, t]$ } \vert Y(x) = i)
\end{gather*}
We define the ``force of transition" or ``transition intensity" between states $i$ and $j$ at age $x$ to be
\begin{gather*}
\mu_x^{ij} = \lim_{h\to0^+} \frac{\hspace{0mm}_hp_x^{ij}}{h} \quad\mbox{ for $i \ne j$}
\end{gather*}
Equivalently,
\begin{gather*}
\hspace{0mm}_hp_x^{ij} = h\mu_x^{ij} + o(h) \quad\mbox{ for $i \ne j$}
\end{gather*}

\textbf{Assumption 8.3} For all states $i$ and $j$ and all ages $x \ge 0$ we assume that $_tp_x^{ij}$ is a differentiable function of $t$.

\color{red}
\textbf{Example 8.3 is foundational and important; I did it but made a few mistakes here and there and missed out on one trick; definitely do this again}
\color{black}

For any State $i$ in a multiple state model with $m+1$ states satisfying the usual assumptions,
\begin{gather*}
\hspace{0mm}_{t+h}p_x^{\overline{ii}} = \hspace{0mm}_tp_x^{\overline{ii}} \hspace{0mm}_hp_{x+t}^{\overline{ii}}, \qquad x, t, h \ge 0
\end{gather*}
For small $h$,
\begin{gather*}
\hspace{0mm}_hp_{x+t}^{\overline{ii}} = 1 - h \sum_{j \ne i} \mu_{x+t}^{ij} + o(h)
\quad\Rightarrow\quad \frac{\hspace{0mm}_{t+h}p_x^{\overline{ii}} - \hspace{0mm}_tp_x^{\overline{ii}}}{h} 
= \hspace{0mm}_tp_x^{\overline{ii}} \left( - \sum_{j\ne i} \mu_{x+t}^{ij} + \frac{o(h)}{h} \right)
\end{gather*}
Taking the limit $h \to 0$,
\begin{gather*}
\frac{d}{dt} \hspace{1mm}_tp_x^{\overline{ii}} = \hspace{0mm}_tp_x^{\overline{ii}} \left( -\sum_{j \ne i} \mu_{x+t}^{ij} \right)
\quad\Rightarrow\quad \hspace{0mm}_tp_x^{\overline{ii}} = \exp\left(-\int_0^t \sum_{j \ne i} \mu_{x+s}^{ij} ds\right)
\end{gather*}

\textbf{Example 8.4} For the model for permanent disability illustrated in Figure 8.3,
\begin{align*}
\hspace{0mm}_{t+h}p_x^{01}
&= \hspace{0mm}_tp_x^{01} \hspace{0mm}_hp_{x+t}^{\overline{11}}
+ \hspace{0mm}_tp_x^{\overline{00}} \hspace{0mm}_hp_{x+t}^{01} \\
&= \hspace{0mm}_tp_x^{01} (1 - h\mu_{x+t}^{12})
+ \hspace{0mm}_tp_x^{\overline{00}} h\mu_{x+t}^{01} + o(h)
\end{align*}
so
\begin{gather*}
\frac{\hspace{0mm}_{t+h}p_x^{01} - \hspace{0mm}_tp_x^{01}}{h}
= -\hspace{0mm}_tp_x^{01} \mu_{x+t}^{12} + \hspace{0mm}_tp_x^{\overline{00}} \mu_{x+t}^{01} + \frac{o(h)}{h}
\end{gather*}
Letting $h \to 0$, 
\begin{gather*}
\frac{d}{dt} \hspace{1mm}_tp_x^{01} = -\hspace{0mm}_tp_x^{01} \mu_{x+t}^{12} + \hspace{0mm}_tp_x^{\overline{00}} \mu_{x+t}^{01}
\end{gather*}
Observe the integrating factor and obtain
\begin{align*}
\frac{d}{dt} \left( \hspace{0mm}_tp_x^{01} \exp \left( \int_0^t \mu_{x+s}^{12} ds \right) \right)
&= \left( \frac{d}{dt} \hspace{1mm}_tp_x^{01} + \hspace{0mm}_tp_x^{01} \mu_{x+t}^{12} \right) \exp \left( \int_0^t \mu_{x+s}^{12} ds \right) \\ &= \exp \left( \int_0^t \mu_{x+s}^{12} ds \right) \hspace{0mm}_tp_x^{\overline{00}} \mu_{x+t}^{01}
\end{align*}
In the case of the permanent disability model where a person starting from the disability state 1 can only either stay put or jump to the death state 2,
\begin{gather*}
\hspace{0mm}_tp_x^{\overline{11}} = \exp \left( - \int_0^t \mu_{x+s}^{12} ds \right) \quad\Rightarrow\quad
\hspace{0mm}_{u-t}p_{x+t}^{\overline{11}} = \exp \left( - \int_t^u \mu_{x+s}^{12} ds \right)
\end{gather*}
\begin{align*}
\frac{d}{dt} \left( \frac{\hspace{0mm}_tp_x^{01}}{\hspace{0mm}_tp_x^{\overline{11}}} \right)
= \hspace{0mm}_tp_x^{\overline{00}} \mu_{x+t}^{01} \exp \left( \int_0^t \mu_{x+s}^{12} ds \right) 
\quad&\Rightarrow\quad
\frac{\hspace{0mm}_up_x^{01}}{\hspace{0mm}_up_x^{\overline{11}}} = \int_0^u \hspace{0mm}_tp_x^{\overline{00}} \mu_{x+t}^{01} \exp \left( \int_0^t \mu_{x+s}^{12} ds \right) dt \\
&\Rightarrow\quad \hspace{0mm}_up_x^{01}
= \int_0^u \hspace{0mm}_tp_x^{\overline{00}} \mu_{x+t}^{01} \exp \left( - \int_t^u \mu_{x+s}^{12} ds \right) dt
\end{align*}

\textbf{7th January 2025, Tuesday, 2.54pm at the 7th floor of the National Library at Bugis}

For any time $t \ge 0$ and states $i, j$ and small $h > 0$, if a life $(x)$ starts in state $i$ at time 0, 
\begin{gather*}
\hspace{0mm}_{t+h}p_x^{ij} 
= \hspace{0mm}_tp_x^{ij} \left( 1 - \sum_{k \ne j} h \mu_{x+t}^{jk} \right) 
+ \sum_{k \ne j} \hspace{0mm}_tp_x^{ik} h \mu_{x+t}^{kj} + o(h)
\end{gather*}
because either the life $(x)$ transitions from state $i$ to state $j$ at time $t$ and then remains in state $j$ for the short time from $t$ to $t+h$, or the life transitions from state $i$ to some other state $k$ at time $t$ and then transitions from that state $k$ to state $j$ during that short time from $t$ to $t+h$. Rearranging and taking the limit as $h \to 0$ from above,
\begin{gather*}
\frac{d}{dt} \hspace{1mm}_tp_x^{ij}
= \lim_{h \to 0^+} \frac{\hspace{0mm}_{t+h}p_x^{ij} - \hspace{0mm}_tp_x^{ij}}{h}
= \sum_{k \ne j} \hspace{0mm}_tp_x^{ik} \mu_{x+t}^{kj} - \sum_{k \ne j} \hspace{0mm}_tp_x^{ij} \mu_{x+t}^{jk}, \qquad i, j \in I
\end{gather*}
These are the celebrated ``forward" equations A. Kolmogorov, a linear system of $|I|^2$ first order ODEs in $|I|^2$ variables $_tp_x^{ij}$ over all $i, j \in I$ with initial condition
\begin{gather*}
_0p_x^{ij} = \delta_{ij}
\end{gather*}

\textbf{Example 8.5} Consider the permanent disability model illustrated in Figure 8.3. 
\begin{enumerate}[label=(\alph*)]
\item Suppose the transition intensities for this model are all constants,
\begin{gather*}
\mu_x^{01} = 0.0279, \qquad \mu_x^{02} = \mu_x^{12} = 0.0229
\end{gather*}
Calculate $_{10}p_{60}^{00}$ and $_{10}p_{60}^{01}$.
\item Now suppose the transition intensities are
\begin{gather*}
\mu_x^{01} = a_1 + b_1 e^{c_1 x}, \qquad \mu_x^{02} = \mu_x^{12} = a_2 + b_2 e^{c_2 x}
\end{gather*}
Calculate $_{10}p_{60}^{00}$ and $_{10}p_{60}^{01}$.
\end{enumerate}

\textbf{Solution 8.5} In the permanent disability model, we cannot return to state 0 once we have left it, and the only transitions out of state 0 are to states 1 and 2, so 
\begin{gather*}
_tp_x^{00} = \hspace{0mm}_tp_x^{\overline{00}} = \exp \left( - \int_0^t \mu_{x+s}^{01} + \mu_{x+s}^{02} ds \right)
\end{gather*}

For the situation (a), this is simply
\begin{gather*}
_{10}p_{60}^{00} = \exp \left( - (0.0279 + 0.0229) \times 10 \right) = e^{-0.508} = 0.601697772
\end{gather*}

For the situation (b), this is
\begin{align*}
_{10}p_{60}^{00} 
&= \exp \left( - \int_{60}^{70} a_1 + b_1 e^{c_1 s} + a_2 + b_2 e^{c_2 s} ds \right) \\
&= \exp \left( - 10 (a_1 + a_2) - \frac{b_1}{c_1} \left( e^{70c_1} - e^{60c_1} \right) - \frac{b_2}{c_2} \left( e^{70c_2} - e^{60c^2} \right) \right)
\\
&= e^{-0.009 - 0.29785603 - 0.231079427} = 0.583952604
\end{align*}

Similarly we cannot return to state 1 once we have left it, and the only transitions out of state 1 are to state 2, so
\begin{gather*}
_tp_x^{11} = \hspace{0mm}_tp_x^{\overline{11}} = \exp \left( - \int_0^t \mu_{x+s}^{12} ds \right)
\end{gather*}

For state 2, there is no leaving so $_tp_x^{22} = 1$. Also,
\begin{gather*}
_tp_x^{10} = \hspace{0mm}_tp_x^{20} = \hspace{0mm}_tp_x^{21} = 0
\end{gather*}

For the other transitions between states, Kolmogorov forward equations are
\begin{gather*}
\frac{d}{dt} \hspace{1mm}_tp_x^{01} 
= \hspace{0mm}_tp_x^{00} \mu_{x+t}^{01}
- \hspace{1mm}_tp_x^{01} \mu_{x+t}^{12}
\end{gather*}
\begin{gather*}
\frac{d}{dt} \hspace{1mm}_tp_x^{02}
= \hspace{0mm}_tp_x^{00} \mu_{x+t}^{02}
+ \hspace{0mm}_tp_x^{01} \mu_{x+t}^{12}
\end{gather*}
\begin{gather*}
\frac{d}{dt} \hspace{1mm}_tp_x^{12}
= \hspace{0mm}_tp_x^{11} \mu_{x+t}^{12}
\end{gather*}

The first equation can be solved using an integrating factor,
\begin{align*}
\frac{d}{dt} \left( \exp \left( \int_0^t \mu_{x+s}^{12} ds \right) \hspace{1mm}_tp_x^{01} \right)
&= \exp \left( \int_0^t \mu_{x+s}^{12} ds \right) \frac{d}{dt} \hspace{1mm}_tp_x^{01}
+ \exp \left( \int_0^t \mu_{x+s}^{12} ds \right) \mu_{x+t}^{12} \hspace{1mm}_tp_x^{01} \\
&= \hspace{1mm}_tp_x^{00} \mu_{x+t}^{01} \exp \left( \int_0^t \mu_{x+s}^{12} ds \right)
\end{align*}
and since $_0p_x^{01} = 0$, 
\begin{gather*}
\exp \left( \int_0^t \mu_{x+s}^{12} ds \right) \hspace{1mm}_tp_x^{01}
= \int_0^t \hspace{1mm}_rp_x^{00} \mu_{x+r}^{01} \exp \left( \int_0^r \mu_{x+s}^{12} ds \right) dr
\end{gather*}
\begin{gather*}
\hspace{0mm}_tp_x^{01}
= \int_0^t \hspace{1mm}_rp_x^{00} \mu_{x+r}^{01} \exp \left( - \int_r^t \mu_{x+s}^{12} ds \right) dr
= \int_0^t \hspace{1mm}_rp_x^{\overline{00}} \mu_{x+r}^{01} \hspace{1mm}_tp_{x+r}^{\overline{11}} dr
\end{gather*}
We have re-derived an earlier formula.

For the situation in (a), this is
\begin{gather*}
\hspace{0mm}_tp_x^{01}
= \int_0^t e^{-(\mu_x^{01} + \mu_x^{02})r} \mu_x^{01} e^{-\mu_x^{12}(t-r)} dr
= \mu_x^{01} e^{-\mu_x^{12} t} \int_0^t e^{-\mu_x^{01} r} dr
= e^{-\mu_x^{12} t} \left( 1 - e^{-\mu_x^{01} t} \right)
\end{gather*}
In particular,
\begin{gather*}
_{10}p_{60}^{01} = e^{-0.229} \left(1 - e^{-0.279} \right) = 0.193630762
\end{gather*}

For the situation in (b), this is
\begin{gather*}
\hspace{0mm}_tp_x^{01}
= \int_0^t 
\exp \left( - \int_x^r a_1 + b_1 e^{c_1 s} + a_2 + b_2 e^{c_2 s} ds \right)
\left(a_1 + b_1 e^{c_1(x+r)} \right)
\exp \left( - \int_r^t a_2 + b_2 e^{c_2 (x+s) } ds \right) dr
\end{gather*}

Insurance benefits are conditional on making a transition into a specified state. For example, a death benefit is payable on transition into the dead state; a permanent disability insurance policy might pay a sum insured on becoming disabled.

Suppose a payment of 1 is made immediately on each future transfer into state $j$ within $n$ years, given that the life is currently in state $i$ (which may be equal to $j$). Then the EPV of the benefit is
\begin{gather*}
\overline{A}_{x:n}^{ij} = \int_0^n \sum_{k \ne j} e^{-\delta t} \hspace{1mm}_tp_x^{ik} \mu_{x+t}^{kj} dt
\end{gather*}
Note that this benefit does not require the transition to be directly from state $i$ to state $j$, and if there is a possibility of multiple transitions into state $j$, then it values a benefit of 1 paid each time the life transitions into state $j$. If there is no definite term to the policy, then we have the whole life function $\overline{A}_x^{ij}$. 

If we need to calculate $\overline{A}_{x:n}^{ij}$ form a set of tables of $\overline{A}_x^{ij}$, we can do so provided that we know probabilities $_np_x^{ij}$. The relationship is
\begin{gather*}
\overline{A}_{x:n}^{ij} = \overline{A}_x^{ij} - e^{-\delta n} \sum_k \hspace{1mm}_np_x^{ik} \overline{A}_{x+n}^{kj}
\end{gather*}

\color{red}
\textbf{I don't understand equation 8.20 in the textbook - look into it again}
\color{black}

We have assume above that the benefit payable on transition from state $i$ to state $j$ is one. Suppose instead that the benefit on any transition to state $j$ is $B_t^{(j)}$ in the next $n$ years. Then the EPV of this new benefit would be
\begin{gather*}
\int_0^n \sum_{k \ne j} B_t^{(j)} e^{-\delta t} \hspace{1mm}_tp_x^{ik} \mu_{x+t}^{kj} dt
\end{gather*}

A ``continuous sojourn annuity" is an annuity payable for the duration of a single stay in a given state. The EPV of a continuous payment of 1 per year paid to a life currently aged $x$ and currently in State $i$, where the payment continues as long as the life remains in State $i$, or until the expiry of the $n$ year term if earlier, is
\begin{gather*}
\overline{a}_{x:n}^{\overline{ii}} = \int_0^n e^{-\delta t} \hspace{1mm}_tp_x^{\overline{ii}} dt
\end{gather*}

We assume that premiums are calculated using the equivalence principle and that lives are in State 0 at the policy inception date. 

\textbf{$n$ regular exercises, $m$ excel based exercises}

\textbf{ALTAM Chapter 9 Multiple Decrement Models} \\
\textbf{9.1 Summary} \\
\textbf{9.2 Examples of multiple decrement models} \\
\textbf{9.3 Actuarial functions for multiple decrement models} \\
\textbf{9.4 Multiple decrement tables} \\
\textbf{9.5 Constructing a multiple decrement table} \\
\textbf{9.6 Comments on multiple decrement notation} \\
\textbf{9.7 Transitions at exact ages}

A multiple decrement model with $m+1$ states is characterised by having a single initial state, and $m$ exit states. We generically refer to the initial state as the active state or State 0. The process makes at most one transition, into one of the exit states. There are no further transitions, so all the exit states are absorbing states, and all the probabilities are of the form $\hspace{0mm}_tp_x^{0j}$ for $j = 0, 1, \dots, m$. We refer to modes of exit as decrements. 

\textbf{$9$ regular exercises, $5$ excel based exercises}

\textbf{ALTAM Chapter 11 Pension Mathematics} (except section 12) (58 pages) \\
\textbf{11.1 Summary} \\
\textbf{11.2 Introduction} \\
\textbf{11.3 The salary scale function} \\
\textbf{11.4 Setting the contribution for a DC plan} \\
\textbf{11.5 The service table} \\
\textbf{11.6 Valuation of final salary plans} \\
\textbf{11.7 Valuing career average earnings plans} \\
\textbf{11.8 Funding the benefits} \\
\textbf{11.9 Projected Unit Credit funding} \\
\textbf{11.10 Traditional Unit Credit funding} \\
\textbf{11.11 Comparing PUC and TUC funding methods} \\
\textbf{11.12 Retiree health benefits} (skip) \\
\textbf{$16$ regular exercises, $4$ excel based exercises}

Two major categories of employer sponsored pension plans are defined contribution and defined benefit.

The defined contribution pension plan specifies how much the employer will contribute, as a percentage of salary, into a plan. The contributions are invested, and the accumulated funds are available to the employee when he or she leaves the company. The contributions may be set to meet a target benefit level, but the actual retirement income may be well below or above the target, depending on the investment experience.

The defined benefit plan specifies a level of benefit, usually in relation to salary near retirement or to salary throughout employment. The contributions from the employer and possibly employee are accumulated to meet the benefit. If the investment or demographic experience is adverse, the contributions can be increased; if the experience is favourable, the contributions may be reduced. The pension plan actuary monitors the plan funding on a regular basis to assess whether the contributions need to be changed. 

The benefit under a DB plan, and the target under a DC plan, are set by consideration of an appropriate ``replacement ratio" or ``replacement rate"
\begin{gather*}
R = \frac{\mbox{pension income in the year after retirement}}{\mbox{salary in the year before retirement}}
\end{gather*}
where we assume the plan member survives the year following retirement. 

The contributions and benefits for most employer sponsored pension plans are related to salaries, so we need to model the progression of salaries through an individual's employment. We use a deterministic model. The rate of salary function $\{\overline{s}_y\}_{y \ge x_0}$ is such that $\overline{s}_y / \overline{s}_x$ is the ratio of the annual rate of salary at age $y$ to the annual rate of salary at age $x$ for any $x, y \ge x_0$. The value of $\overline{s}_{x_0}$ can be set arbitrarily to any positive number. 

\textbf{Example 11.1} Consider an employee aged 30 whose current annual salary rate is \$30,000 and assume she will still be employed at exact age 41. 
\begin{enumerate}[label = (\alph*)]
\item Suppose the employee's rate of salary function $\{\overline{s}_y\}_{y \ge 20}$ is given by
\begin{gather*}
\overline{s}_y = 1.04^{y-20}
\end{gather*}
\begin{enumerate}[label=(\roman*)]
\item Calculate her annual rate of salary at exact age 30.5
\item Calculate her salary for the year of age 30 to 31
\item Calculate her annual rate of salary at exact age 40.5
\item Calculate her salary for the year of age 40 to 41
\end{enumerate}
\item Now suppose that each year the rate of salary increases by 4\% three months after an employee's birthday and then remains constant for a year. Repeat the above calculations.
\end{enumerate}

\textbf{Solution 11.1} For the situation in part (a),
\begin{gather*}
\mbox{salary rate at age 30.5} 
= \frac{\overline{s}_{30.5}}{\overline{s}_{30}} \times 30,000
= 1.04^{0.5} \times 30,000 = 30594.1170816
\end{gather*}
\begin{gather*}
y = a^x \quad\Rightarrow\quad \log y = x \log a \quad\Rightarrow\quad \frac{1}{y} \frac{dy}{dx} = \log a
\quad\Rightarrow\quad \frac{dy}{dx} = a^x \log a \quad\Rightarrow\quad \int_r^s a^x dx = \frac{1}{\log a} \left( a^s - a^r \right)
\end{gather*}
\begin{gather*}
\mbox{salary for the year between 30 to 31} = \frac{1.04 - 1}{\log 1.04} \times 30,000
= 30596.0780292
\end{gather*}
\begin{gather*}
\mbox{salary rate at age 40.5} 
= \frac{\overline{s}_{40.5}}{\overline{s}_{30}} \times 30,000
= 1.04^{10.5} \times 30,000 = 45286.7669621
\end{gather*}
\begin{gather*}
\mbox{salary for the year between 40 to 41} = \frac{1.04^{11} - 1.04^{10}}{\log 1.04} \times 30,000
= 45289.6696437
\end{gather*}

For the situation in part (b),
\begin{gather*}
\mbox{salary rate at age 30.5} = 1.04 \times 30,000 = 31200
\end{gather*}
\begin{gather*}
\mbox{salary for the year between 30 to 31} = 0.25 \times 30,000 + 0.75 \times 31200 = 30900
\end{gather*}
\begin{gather*}
\mbox{salary rate at age 40.5} = 1.04^{11} \times 30,000 = 46183.6216895
\end{gather*}
\begin{gather*}
\mbox{salary for the year between 40 to 41} = 0.25 \times 1.04^{10} \times 30,000 + 0.75 \times 1.04^{11} \times 30,000
= 45739.548404
\end{gather*}

\color{red}
\textbf{My answers for (b)(iii) and (b)(iv) disagree with the textbook's. I think they are wrong.}
\color{black}

In practice it is common to model the progression of salaries using a salary scale $\{s_y\}_{y \ge x_0}$ where the value $s_{x_0}$ can be set arbitrarily as any positive number, and for $x, y \ge x_0$,
\begin{gather*}
\frac{s_y}{s_x} = \frac{\mbox{salary received in year of age $y$ to $y+1$}}{\mbox{salary received in year of age $x$ to $x+1$}}
= \frac{\int_0^1 \overline{s}_{y+t} dt}{\int_0^1 \overline{s}_{x+t} dt}
\end{gather*}
where we assume the individual remains in employment throughout the period from age $x$ to age $y+1$. 

\newpage

\begin{center}
\textbf{FAM Syllabus (brief)}
\end{center}

\textbf{FAM Chapter 1 Introduction to Life and Long-Term Health Insurance} (33 pages) \\
\textbf{FAM Chapter 2 Survival Models} (24 pages) \\
\textbf{FAM Chapter 3 Life Tables and Selection} (except sections 4, 10, 11, 12) (46 pages) \\
\textbf{FAM Chapter 4 Insurance Benefits} (37 pages) \\
\textbf{FAM Chapter 5 Annuities} (38 pages) \\
\textbf{FAM Chapter 6 Premium Calculation} (39 pages) \\
\textbf{FAM Chapter 7 Policy Values} (sections 1-3 except 2.4, 2.5 plus sections 7, 8) (68 pages) \\
\textbf{FAM Chapter 16 Option Pricing}

\begin{center}
\textbf{ALTAM Syllabus (brief)}
\end{center}

\textbf{ALTAM Chapter 3 Life Tables and Selection} (sections 4 and 10) (46 pages) \\
\textbf{ALTAM Chapter 7 Policy Values} (sections 2.4 and 4) (68 pages) \\
\textbf{ALTAM Chapter 8 Multiple State Models} (73 pages) \\
\textbf{ALTAM Chapter 9 Multiple Decrement Models} (29 pages) \\
\textbf{ALTAM Chapter 10 Joint Life and Last Survivor Benefits} (35 pages) \\
\textbf{ALTAM Chapter 11 Pension Mathematics} (except section 12) (58 pages) \\
\textbf{ALTAM Chapter 13 Emerging Costs for Traditional Life Insurance} (except section 8) (39 pages) \\
\textbf{ALTAM Chapter 14 Universal Life Insurance} (26 pages) \\
\textbf{ALTAM Chapter 15 Emerging Costs for Equity-Linked Insurance} (sections 1-3) (29 pages) \\
\textbf{ALTAM Chapter 17 Embedded Options} (31 pages) \\
\textbf{ALTAM Chapter 18 Estimating Survival Models} (sections 5 and 6 note sections 1-3 are background only) (36 pages)

\newpage

\begin{center}
\textbf{FAM Syllabus (detailed)}
\end{center}

\textbf{FAM Chapter 1 Introduction to Life and Long-Term Health Insurance} (33 pages) \\
\textbf{1.1 Summary} \\
\textbf{$m$ regular exercises, $n$ excel based exercises}

\textbf{FAM Chapter 2 Survival Models} (24 pages) \\
\textbf{2.1 Summary} \\
\textbf{$m$ regular exercises, $n$ excel based exercises}

\textbf{FAM Chapter 3 Life Tables and Selection} (except sections 4, 10, 11, 12) (46 pages) \\
\textbf{3.1 Summary} \\
\textbf{$m$ regular exercises, $n$ excel based exercises}

\textbf{FAM Chapter 4 Insurance Benefits} (37 pages) \\
\textbf{4.1 Summary} \\
\textbf{$m$ regular exercises, $n$ excel based exercises}

\textbf{FAM Chapter 5 Annuities} (38 pages) \\
\textbf{5.1 Summary} \\
\textbf{$m$ regular exercises, $n$ excel based exercises}

\textbf{FAM Chapter 6 Premium Calculation} (39 pages) \\
\textbf{6.1 Summary} \\
\textbf{6.2 Preliminaries} \\
\textbf{6.3 The loss at issue random variable} \\
\textbf{6.4 The equivalence principle premium} \\
\textbf{6.5 Profit} \\
\textbf{6.6 The portfolio percentile premium principle} \\
\textbf{6.7 Extra risks} \\
\textbf{$22$ regular exercises, $3$ excel based exercises}

\textbf{FAM Chapter 7 Policy Values} (sections 1-3 except 2.4, 2.5 plus sections 7, 8) (68 pages) \\
\textbf{7.1 Summary} \\
\textbf{7.2 Policies with annual cash flows} \\
\textbf{7.3 Policy values for policies with cash flows at $1/m$thly intervals} \\
\textbf{7.4 Policy values with continuous cash flows} (skip) \\
\textbf{7.5 Policy alterations} (skip) \\
\textbf{7.6 Retrospective policy values} (skip) \\
\textbf{7.7 Negative policy values} \\
\textbf{7.8 Deferred acquisition expenses and modified net premium reserves} \\
\textbf{7.9 Other reserves} (skip) \\
\textbf{$19$ regular exercises, $5$ excel based exercises}

\textbf{FAM Chapter 16 Option Pricing} \\
\textbf{16.1 Summary} \\
\textbf{$m$ regular exercises, $n$ excel based exercises}

\newpage

\begin{center}
\textbf{ALTAM Syllabus (detailed)}
\end{center}

\textbf{ALTAM Chapter 3 Life Tables and Selection} (sections 4 and 10) \\
\textbf{ALTAM Chapter 7 Policy Values} (sections 2.4 and 4)

\textbf{ALTAM Chapter 8 Multiple State Models} (73 pages) \\
\textbf{8.1 Summary} \\
\textbf{8.2 Examples of multiple state models} \\
\textbf{8.3 Assumptions and notation} \\
\textbf{8.4 Formulae for probabilities} \\
\textbf{8.5 Numerical evaluation of probabilities} \\
\textbf{8.6 State-dependent insurance and annuity functions} \\
\textbf{8.7 Premiums} \\
\textbf{8.8 Policy values} \\
\textbf{8.9 Applications of multiple state models in long-term health and disability insurance} \\
\textbf{8.10 Markov multiple state models in discrete time} \\
\textbf{$20$ regular exercises, $4$ excel based exercises}

\textbf{ALTAM Chapter 9 Multiple Decrement Models} (29 pages) \\
\textbf{9.1 Summary} \\
\textbf{9.2 Examples of multiple decrement models} \\
\textbf{9.3 Actuarial functions for multiple decrement models} \\
\textbf{9.4 Multiple decrement tables} \\
\textbf{9.5 Constructing a multiple decrement table} \\
\textbf{9.6 Comments on multiple decrement notation} \\
\textbf{9.7 Transitions at exact ages} \\
\textbf{$9$ regular exercises, $5$ excel based exercises}

\textbf{ALTAM Chapter 10 Joint Life and Last Survivor Benefits} (35 pages) \\
\textbf{10.1 Summary} \\
\textbf{10.2 Joint life and last survivor benefits} \\
\textbf{10.3 Joint life notation} \\
\textbf{10.4 Independent future lifetimes} \\
\textbf{10.5 A multiple state model for independent future lifetimes} \\
\textbf{10.6 A model with dependent future lifetimes} \\
\textbf{10.7 The common shock model} \\
\textbf{$16$ regular exercises, $5$ excel based exercises}

\textbf{ALTAM Chapter 11 Pension Mathematics} (except section 12) (58 pages) \\
\textbf{11.1 Summary} \\
\textbf{11.2 Introduction} \\
\textbf{11.3 The salary scale function} \\
\textbf{11.4 Setting the contribution for a DC plan} \\
\textbf{11.5 The service table} \\
\textbf{11.6 Valuation of final salary plans} \\
\textbf{11.7 Valuing career average earnings plans} \\
\textbf{11.8 Funding the benefits} \\
\textbf{11.9 Projected Unit Credit funding} \\
\textbf{11.10 Traditional Unit Credit funding} \\
\textbf{11.11 Comparing PUC and TUC funding methods} \\
\textbf{11.12 Retiree health benefits} (skip) \\
\textbf{$16$ regular exercises, $4$ excel based exercises}

\textbf{ALTAM Chapter 13 Emerging Costs for Traditional Life Insurance} (except section 8) (39 pages) \\
\textbf{13.1 Summary} \\
\textbf{13.2 Introduction} \\
\textbf{13.3 Profit testing a term insurance policy} \\
\textbf{13.4 Profit testing principles} \\
\textbf{13.5 Profit measures} \\
\textbf{13.6 Using the profit test to calculate the premium} \\
\textbf{13.7 Using the profit test to calculate reserves} \\
\textbf{13.8 Profit testing for participating insurance} (skip) \\
\textbf{13.9 Profit testing for multiple state-dependent insurance} \\
\textbf{$9$ regular exercises, $7$ excel based exercises}

\textbf{ALTAM Chapter 14 Universal Life Insurance} (26 pages) \\
\textbf{14.1 Summary} \\
\textbf{14.2 Introduction} \\
\textbf{14.3 Universal life insurance} \\
\textbf{$8$ regular exercises, $2$ excel based exercises}

\textbf{ALTAM Chapter 15 Emerging Costs for Equity-Linked Insurance} (sections 1-3) (29 pages) \\
\textbf{15.1 Summary} \\
\textbf{15.2 Equity-linked insurance} \\
\textbf{15.3 Deterministic profit testing for equity-linked insurance} \\
\textbf{15.4 Stochastic profit testing} (skip) \\
\textbf{15.5 Stochastic pricing} (skip) \\
\textbf{15.6 Stochastic reserving} (skip) \\
\textbf{$5$ regular exercises, $0$ excel based exercises}

\textbf{ALTAM Chapter 17 Embedded Options} (31 pages) \\
\textbf{17.1 Summary} \\
\textbf{17.2 Introduction} \\
\textbf{17.3 Guaranteed minimum maturity benefit} \\
\textbf{17.4 Guaranteed minimum death benefit} \\
\textbf{17.5 Funding methods for embedded options} \\
\textbf{17.6 Risk management} \\
\textbf{17.7 Profit testing} \\
\textbf{$2$ regular exercises, $3$ excel based exercises}

\textbf{ALTAM Chapter 18 Estimating Survival Models} (sections 5 and 6 note sections 1-3 are background only) (36 pages) \\
\textbf{18.1 Summary} \\
\textbf{18.2 Introduction} \\
\textbf{18.3 Actuarial lifetime data} \\
\textbf{18.4 Non-parametric survival function estimation} (skip) \\
\textbf{18.5 The alive-dead model} \\
\textbf{18.6 Estimation of transition intensities in multiple state models} \\
\textbf{$m$ regular exercises, $n$ excel based exercises}

\newpage

\begin{center}
\textbf{ALTAM Chapter 8 Multiple State Models} (73 pages)
\end{center}

\textbf{8.1 Summary} \\
\textbf{8.2 Examples of multiple state models} \\
\textbf{8.3 Assumptions and notation} \\
\textbf{8.4 Formulae for probabilities} \\
\textbf{8.5 Numerical evaluation of probabilities} \\
\textbf{8.6 State-dependent insurance and annuity functions} \\
\textbf{8.7 Premiums} \\
\textbf{8.8 Policy values} \\
\textbf{8.9 Applications of multiple state models in long-term health and disability insurance} \\
\textbf{8.10 Markov multiple state models in discrete time} \\
\textbf{$20$ regular exercises, $4$ excel based exercises}

Figure 8.1 The alive-dead model (pg 287)
\begin{gather*}
\mbox{state 0 (alive)} \quad\longrightarrow\quad \mbox{state 1 (dead)}
\end{gather*}

Figure 8.2 The accidental death model (pg 288)
\begin{gather*}
\mbox{state 2 (dead - other causes)} \quad \longleftarrow \quad \mbox{state 0 (alive)} \quad \longrightarrow \quad \mbox{state 1 (dead - accident)}
\end{gather*}

Figure 8.3 The permanent disability model (pg 289)
\[\begin{tikzcd}
\mbox{state 0 (healthy)} \arrow{rr} \arrow{dr} & & \mbox{state 1 (disabled)} \arrow{dl} \\
& \mbox{state 2 (dead)} &
\end{tikzcd}\]

Figure 8.4 The sickness-death model for disability income insurance (pg 290)
\[\begin{tikzcd}
\mbox{state 0 (healthy)} \arrow{rr} \arrow{dr} & & \mbox{state 1 (sick)} \arrow{ll} \arrow{dl} \\
& \mbox{state 2 (dead)} &
\end{tikzcd}\]

\textbf{Assumption 8.1 (Markov property)} For any states $i, j$ and times $t, t + s$, the conditional probability
\begin{gather*}
\mathbb{P}(Y(t + s) = j \vert Y(t) = i)
\end{gather*}
is well-defined, i.e. its value does not depend on any information about the process before time $t$.

\textbf{Assumption 8.2} For any positive interval of time $h$, 
\begin{gather*} 
\mathbb{P}(\mbox{2 or more transitions within a time period of length $h$}) = o(h)
\end{gather*}

\textbf{Notation} For states $i$ and $j$ in a multiple state model and for $x, t \ge 0$, we define
\begin{gather*}
\hspace{0mm}_tp_x^{ij} = \mathbb{P}(Y(x+t) = j \vert Y(x) = i), \qquad
\hspace{0mm}_tp_x^{\overline{ii}} = \mathbb{P}(Y(x+s) = i \mbox{ for all } s\in [0, t] \vert Y(x) = i)
\end{gather*}
The \textbf{force of transition} or \textbf{transition intensity} between states $i$ and $j$ at age $x$ is
\begin{gather*}
\mu_x^{ij} = \lim_{h \to 0^+} \frac{\hspace{0mm}_hp_x^{ij}}{h} \quad\mbox{ for }\quad i \ne j
\end{gather*}

\textbf{Assumption 8.3} For all states $i$ and $j$ and all ages $x \ge 0$, we assume that $\hspace{0mm}_tp_x^{ij}$ is a differentiable function of $t$.

For a general multiple state model and small $h > 0$,
\begin{gather*}
\hspace{0mm}_hp_x^{\overline{ii}} = 1 - h \sum_{j \ne i} \mu_x^{ij} + o(h) \tag{8.7}
\end{gather*}

For any state $i$ in a multiple state model satisfying assumptions 8.1 to 8.3,
\begin{gather*}
\hspace{0mm}_tp_x^{\overline{ii}} = \exp \left( - \int_0^t \sum_{j \ne i} \mu_{x+s}^{ij} ds \right) \tag{8.9}
\end{gather*}

\textbf{8.4.1 Kolmogorov's forward equations} (pg 300)
\begin{gather*}
\frac{d}{dt} \hspace{0mm}_tp_x^{ij} = \sum_{k \ne j} \left( \hspace{1mm}_tp_x^{ik} \mu_{x+t}^{kj} - \hspace{1mm}_tp_x^{ij} \mu_{x+t}^{jk} \right) \tag{8.16}
\end{gather*}

Suppose a payment of 1 is made immediately on each future transfer into state $j$ within $n$ years, given that the life is currently in state $i$ (which may be equal to $j$). Then the EPV of the benefit is
\begin{gather*}
\overline{A}_{x:n}^{ij} = \int_0^n \sum_{k \ne j} e^{-\delta t} \hspace{1mm}_tp_x^{ik} \mu_{x+t}^{kj} dt \tag{8.18}
\end{gather*}

If we need to calculate $\overline{A}_{x:n}^{ij}$ from a set of tables of $\overline{A}_x^{ij}$, we can do so provided that we know the probabilities $\hspace{0mm}_np_x^{ij}$. The relationship is
\begin{gather*}
\overline{A}_{x:n}^{ij} = \overline{A}_x^{ij} - e^{-\delta n} \sum_{k \ne j} \hspace{1mm}_np_x^{ik} \overline{A}_{x+n}^{kj} \tag{8.19}
\end{gather*}

More generally, if the benefit on any transition to state $j$ is $B_t^{(j)}$ in the next $n$ years, then
\begin{gather*}
\int_0^n \sum_{k \ne j} B_t^{(j)} e^{-\delta t} \hspace{1mm}_tp_x^{ik} \mu_{x+t}^{kj} dt \tag{8.21}
\end{gather*}

Figure 8.6 Four state permanent disability model (pg 308)
\[\begin{tikzcd}
\mbox{state 0 (healthy)} \arrow{rr} \arrow{dd} && \mbox{state 1 (disabled)} \arrow{dd} \\ \\
\mbox{state 2 (dead)} && \mbox{state 3 (dead)}
\end{tikzcd}\]

Suppose we have a life aged $x$ currently in state $i$ of a multiple state model. The EPV of an annuity of 1 per year payable continuously throughout $(x)$'s lifetime, but only while $(x)$ is some specified state $j$ (which may be equal to $i$), is
\begin{gather*}
\overline{a}_x^{ij} = \int_0^{\infty} e^{-\delta t} \hspace{1mm}_tp_x^{ij} dt \tag{8.22}
\end{gather*}

For annuities paid at discrete times, the EPV of an annuity of 1 per year payable at the start of each year for a maximum of $n$ years, conditional on the life being in state $j$ at the payment date, is
\begin{gather*}
\ddot{a}_{x:n}^{ij} = \hspace{1mm}_0p_x^{ij} + \nu \hspace{1mm}_1p_x^{ij} + \nu^2 \hspace{1mm}_2p_x^{ij} + \dots
+ \nu^{n-1} \hspace{1mm}_{n-1}p_x^{ij} = \sum_{k = 0}^{n-1} \nu^k \hspace{1mm}_kp_x^{ij}
\end{gather*}

In situations where we have a set of tables giving the EPV of a lifetime state-dependent annuity, we can use these tables to obtain the EPV of a state-dependent $n$-year term annuity, provided that we know the probabilities $\hspace{0mm}_np_x^{ij}$ via
\begin{gather*}
\overline{a}_{x:n}^{ij} = \overline{a}_x^{ij} - e^{-\delta n} \sum_{k} \hspace{1mm}_np_x^{ik} \overline{a}_{x+n}^{kj} \tag{8.23}
\end{gather*}

Recall from section 5.11.3 the Euler-Maclaurin numerical integration formula
\begin{gather*}
\int_0^{\infty} g(t) dt = h \sum_{k = 0}^{\infty} g(kh) - \frac{h}{2} g(0) + \frac{h^2}{12} g'(0) - \frac{h^4}{720} g'''(0) + \dots \tag{8.25}
\end{gather*}

\textbf{Two-term Woolhouse approximations} (pg 311)
\begin{gather*}
\ddot{a}_x^{(m)ij} \approx \overline{a}_x^{ij} \quad\mbox{ for }\quad i \ne j \tag{8.26}
\end{gather*}
\begin{gather*}
\ddot{a}_x^{(m)ii} \approx \overline{a}_x^{ii} + \frac{1}{2m} \tag{8.27}
\end{gather*}

\textbf{Three-term Woolhouse approximations} (pg 311)
\begin{gather*}
\ddot{a}_x^{(m)ij} \approx \overline{a}_x^{ij} - \frac{\mu_x^{ij}}{12m^2} \quad\mbox{ for }\quad i \ne j \tag{8.28}
\end{gather*}
\begin{gather*}
\ddot{a}_x^{(m)ii} \approx \overline{a}_x^{ii} + \frac{1}{2m} + \frac{1}{12m^2} \left( \mu_x^{i \bullet} + \delta \right)
\quad\mbox{ where }\quad \mu_x^{i \bullet} = \sum_{k \ne i} \mu_x^{ik} \tag{8.29}
\end{gather*}

A \textbf{continuous sojourn annuity} is an annuity payable for the duration of a single stay in a given state. The EPV of a continuous payment of 1 per year paid to a life currently aged $x$ and currently in state $i$, where the payment continues as long as the life remains in state $i$, or until the expiry of the $n$ year term if earlier, is
\begin{gather*}
\overline{a}_{x:n}^{\overline{ii}} = \int_0^n \hspace{1mm}_tp_x^{\overline{ii}} e^{-\delta t} dt \tag{8.30}
\end{gather*}

When $i \ne j$, the general form for $\overline{a}_x^{ij}$ using continuous sojourn annuity EPVs is
\begin{gather*}
\overline{a}_x^{ij} = \int_0^{\infty} e^{-\delta t} a^{\overline{jj}}_{x+t} \sum_{k \ne j} \hspace{1mm}_tp_x^{ik} \mu_{x+t}^{kj} dt
\tag{8.33}
\end{gather*}
For the case $i = j$, we have
\begin{gather*}
\overline{a}_x^{ii}
= \overline{a}_x^{\overline{ii}} + \int_0^{\infty} e^{-\delta t} \overline{a}_{x+t}^{\overline{ii}} \sum_{k \ne i} \hspace{1mm}_tp_x^{ik} \mu_{x+t}^{ki} dt
\end{gather*}

The \textbf{state $j$ policy value at time $t$} is the expected value of the future loss random variable for a policy which is in state $j$ at time $t$,
\begin{gather*}
\hspace{0mm}_tV^{(j)}
= \mbox{ EPV at time $t$ of future benefits + expenses }
- \mbox{ EPV at time $t$ of future premiums}
\end{gather*}
given that the insured is in state $j$ at time $t$. 

\newpage

\textbf{10th January 2025, Friday, 3.38pm at the 7th floor of the National Library at Bugis}

\textbf{Theorem 4.3.2} (Jensen's inequality) Let $X$ be an integrable random variable with values in $I$ and let $c : I \to \mathbb{R}$ be convex. Then $\mathbb{E}(c(X))$ is well defined and
\begin{gather*}
\mathbb{E}(c(X)) \ge c(\mathbb{E}(X))
\end{gather*}

\textbf{Theorem 4.4.1} (H{\"o}lder's inequality) Let $p, q \in (1, \infty)$ be conjugate indices. Then for all measurable functions $f$ and $g$, we have
\begin{gather*}
\mu(\abs{fg}) \le \norm{f}_p \norm{g}_q
\end{gather*}

\textbf{Theorem 4.4.2} (Minkowski's inequality) For $p \in [1, \infty)$ and measurable functions $f$ and $g$, we have
\begin{gather*}
\norm{f + g}_p \le \norm{f}_p + \norm{g}_p
\end{gather*}

\textbf{Theorem 4.5.1} Let $\mathcal{A}$ be a $\pi$-system on $E$ generating $\mathcal{e}$ with $\mu(A) < \infty$ for all $A \in \mathcal{A}$, and such that $E_n \uparrow E$ for some $(E_n : n \in \mathbb{N})$ in $\mathcal{A}$. Define
\begin{gather*}
V_0 = \left\{ \sum_{k = 1}^n a_k 1_{A_k} : a_k \in \mathbb{R}, \hspace{1mm} A_k \in \mathcal{A}, \hspace{1mm} n \in \mathbb{N} \right\}
\end{gather*}
Let $p \in [1, \infty)$. Then $V_0 \subseteq L^p$. Moreover for all $f \in L^p$ and all $\epsilon > 0$ there exists $v \in V_0$ such that $\norm{v - f}_p \le \epsilon$. 

\textbf{Theorem 5.1.1} (Completeness of $L^p$) Let $p \in [1, \infty]$. Let $(f_n : n \in \mathbb{N})$ be a sequence in $L^p$ such that
\begin{gather*}
\norm{f_n - f_m}_p \to 0 \quad\mbox{ as }\quad m, n \to \infty
\end{gather*}
Then there exists $f \in L^p$ such that
\begin{gather*}
\norm{f_n - f}_p \to 0 \quad\mbox{ as }\quad n \to \infty
\end{gather*}

\textbf{Theorem 5.2.1} (Orthogonal projection) Let $V$ be a closed subspace of $L^2$. Then each $f \in L^2$ has a decomposition $f = v + u$ with $v \in V$ and $u \in V^{\perp}$. Moreover $\norm{f - v}_2 \le \norm{f - g}_2$ for all $g \in V$, with equality only if $g = v$ a.e.

\textbf{Theorem 6.1.1} (Bounded convergence) Let $(X_n : n \in \mathbb{N})$ be a sequence of random variables with $X_n \to X$ in probability and $\abs{X_n} \le C$ for all $n$, for some constant $C < \infty$. Then $X_n \to X$ in $L^1$. 

\textbf{Theorem 6.2.3} Let $X$ be a random variable and let $(X_n : n \in \mathbb{N})$ be a sequence of random variables. The following are equivalent:
\begin{enumerate}[label = (\alph*)]
\item $X_n \in L^1$ for all $n$, $X \in L^1$ and $X_n \to X$ in $L^1$
\item $\{X_n : n \in \mathbb{N}\}$ is UI and $X_n \to X$ in probability
\end{enumerate}

\textbf{Theorem 7.7.1} Let $X$ be a random variable in $\mathbb{R}^d$. Then the distribution $\mu_X$ of $X$ is uniquely determined by its characteristic function $\phi_X$. Moreover, in the case where $\phi_X$ is integrable, $\mu_X$ has a continuous bounded density function given by
\begin{gather*}
f_X(x) = \frac{1}{(2\pi)^d} \int_{\mathbb{R}^d} \phi_X(u) e^{-i\braket{u, x}} du
\end{gather*}
Moreover, if $(X_n : n \in \mathbb{N})$ is a sequence of random variables in $\mathbb{R}^d$ such that $\phi_{X_n}(u) \to \phi_X(u)$ as $n \to \infty$ for all $u \in \mathbb{R}^d$, then $X_n$ converges weakly to $X$. 

\textbf{Theorem 10.2.2} (Strong law of large numbers) Let $(Y_n : n \in \mathbb{N})$ be a sequence of independent identically distributed integrable random variables with mean $\nu$. Set $S_n = Y_1 + \dots + Y_n$. Then
\begin{gather*}
S_n / n \to \nu \quad\mbox{ a.s. }\quad \mbox{ as $n \to \infty$ }
\end{gather*}

\textbf{Theorem 10.3.1} (Central limit theorem) Let $(X_n : n \in \mathbb{N})$ be a sequence of independent identically distributed random variables with mean 0 and variance 1. Set $S_n = X_1 + \dots + X_n$. Then for all $x \in \mathbb{R}$, as $n \to \infty$,
\begin{gather*}
\mathbb{P} \left( \frac{S_n}{\sqrt{n}} \le x \right) \to \frac{1}{\sqrt{2\pi}} \int_0^x e^{-\frac{y^2}{2}} dy
\end{gather*}

\textbf{Theorem 2.2.2} (Optional stopping theorem) Let $X$ be a supermartingale and let $S$ and $T$ be bounded stopping times with $S \le T$. Then $\mathbb{E}(X_T) \le \mathbb{E}(X_S)$. 

\textbf{Proposition} Let $M \in \mathcal{M}^2_c$ and $H \in \mathcal{E}$. Then $H \cdot M \in \mathcal{M}^2_c$ and 
\begin{gather*}
\norm{H \cdot M}_{\mathcal{M}^2} = \mathbb{E} \left( \int_0^{\infty} H_s^2 \hspace{1mm} d\braket{M}_s \right)
\end{gather*}

\textbf{Proposition} Let $M \in \mathcal{M}^2_c$ and $H \in \mathcal{E}$. Then
\begin{gather*}
\braket{H \cdot M, N} = H \cdot \braket{M, N} \qquad \forall \quad N \in \mathcal{M}^2_c
\end{gather*}

\textbf{Fact} $L^2(M)$ is a Hilbert space.

\textbf{Proposition} Let $M \in \mathcal{M}^2_c$. Then $\mathcal{E}$ is dense in $L^2(M)$.

\textbf{Proposition} Let $M \in \mathcal{M}^2_c$. Then
\begin{enumerate}[label = (\roman*)]
\item The map $H \in \mathcal{E} \mapsto H \cdot M \in \mathcal{M}^2_c$ extends uniquely to an isometry $L^2(M) \to \mathcal{M}^2_c$, the Ito isometry
\item $H \cdot M$ is the unique martingale in $\mathcal{M}^2_c$ s.t.
\begin{gather*}
\braket{H \cdot M, N} = H \cdot \braket{M, N} \qquad \forall \quad N \in \mathcal{M}^2_c
\end{gather*}
\end{enumerate}

\textbf{Theorem} Let $M$ be a continuous local martingale. For eery $H \in L^2_{loc}(m)$ there is a unique continuous local martingale $H \cdot M$ with $(H \cdot M)_0 = 0$ s.t.
\begin{gather*}
\braket{H \cdot M, N} = H \cdot \braket{M, N} \qquad \forall \quad N \mbox{ continuous local martingale }
\end{gather*}
If $M \in \mathcal{M}^2_c$ and $H \in L^2(M)$ then the definition of $H \cdot M$ is the same as previously.

\textbf{Corollary} Let $X$ be a continuous semimartingale, and let $H$ be a locally bounded adapted left-continuous process. Then for any sequence of subdivisions $0 = t_0^{(m)} < \dots < t_{n_m}^{(m)} = t$ of $[0, t]$ with $\max_i \abs{t_i^{(m)} - t_{i-1}^{(m)}} \to 0$,
\begin{gather*}
\lim_{m \to \infty} \sum_{i = 1}^{n_m} H_{t{i-1}^{(m)}} \left( X_{t_i^{(m)}} - X_{t_{i-1}^{(m)}} \right) = \int_0^t H_s dX_s
\end{gather*}
uniformly on compact sets in probability. 

Unlike the case that $X$ is of finite variation, it is here essential that $H$ is evaluated at the left endpoint of the interval $(t_{i-1}^{(m)}, t_i^{(m)}]$. The choice of the left endpoint gives the Ito integral. Choosing the average of the left and right endpoints gives the Stratonovich integral, which is generally not a martingale. 

\textbf{Theorem} (Integration by parts) Let $X, Y$ be continuous semimartingales. Then a.s. for all $t$,
\begin{gather*}
X_t Y_t - X_0 Y_0 = \int_0^t X_s dY_s + \int_0^t Y_s dX_s + \braket{X, Y}_t
\end{gather*}

\textbf{Theorem} (Ito's formula) Let $X^1, \dots, X^p$ be continuous semimartingales, and let $f : \mathbb{R}^p \to \mathbb{R}$ be $C^2$. Then a.s.
\begin{gather*}
f(X_t) = f(X_0) + \sum_{i = 1}^p \int_0^t \frac{\partial f}{\partial X^i}(X_s) \hspace{1mm} dX^i_s + \frac{1}{2} \sum_{i, j = 1}^p \int_0^t \frac{\partial^2 f}{\partial X^i \partial X^j}(X_s) \hspace{1mm} d\braket{X^i, X^j}_s
\end{gather*}

\newpage

Chin - Problems and Solutions in Mathematical Finance

\begin{enumerate}

\item General Probability Theory (47 questions)

\begin{enumerate} 
\item Probability spaces (13 questions)
\item Discrete and continuous random variables (19 questions)
\item Properties of expectations (15 questions)
\end{enumerate}

\item Wiener Process (39 questions)

\begin{enumerate}
\item Basic properties (13 questions)
\item Markov property (3 questions)
\item Martingale property (7 questions)
\item First passage time (7 questions)
\item Reflection principle (5 questions) 
\item Quadratic variation (4 questions)
\end{enumerate}

\item Stochastic Differential Equations (53 questions)

\begin{enumerate}
\item Ito calculus (16 questions)
\item One-dimensional diffusion process (24 questions)
\item Multi-dimensional diffusion process (13 questions)
\end{enumerate}

\item Change of Measure (36 questions)

\begin{enumerate} 
\item Martingale representation theorem (2 questions)
\item Girsanov's theorem (17 questions)
\item Risk-neutral measure (17 questions)
\end{enumerate}

\item Poisson Process (47 questions)

\begin{enumerate} 
\item Properties of Poisson process (25 questions)
\item Jump diffusion process (6 questions)
\item Girsanov's theorem for jump processes (12 questions)
\item Risk-neutral measure for jump processes (4 questions)
\end{enumerate}

\end{enumerate}

\newpage

\begin{center}
\textbf{QFIQF}
\end{center}

Fall 2020, 2021, 2022 and Spring 2021, 2022 have 15 questions each, total 100 points

Spring 2023, 2024 have 11 questions each and Fall 2023 has 10 questions, total 70 points

Total 107 past year questions to practice on so about 2.5 questions per day on average over 40 days

\begin{center}
\textbf{1. Stochastic calculus}
\end{center}

Hirsa \& Neftci - An Introduction to the Mathematics of Financial Derivatives - Chapters 1 to 15

\textbf{01 Financial mathematics a brief introduction} (6 questions, 11 pages) \\
\textbf{02 A primer on the arbitrage theorem} (9 questions, 20 pages) \\
\textbf{03 Review of deterministic calculus} (12 questions, 21 pages) \\
\textbf{04 Pricing derivatives models and notation} (4 questions, 10 pages) \\
\textbf{05 Tools in probability theory} (6 questions, 22 pages) \\
\textbf{06 Martingales and martingale representation} (7 questions, 24 pages) \\
\textbf{07 Differentiation in stochastic environments} (3 questions, 12 pages) \\
\textbf{08 Wiener process, Levy processes and rare events in financial markets} (4 questions, 22 pages) \\
\textbf{09 Integration in stochastic environments} (10 questions, 18 pages) \\
\textbf{10 Ito's lemma} (5 questions, 16 pages) \\

\textbf{11 The dynamics of derivative prices} (6 questions, 18 pages) \\
\textbf{12 Pricing derivative products - partial differential equations} (6 questions, 18 pages) \\
\textbf{13 PDEs and PIDEs - an application} (5 questions, 15 pages) \\
\textbf{14 Pricing derivative products - equivalent martingale measures} (5 questions, 22 pages) \\
\textbf{15 Equivalent martingale measures} (4 questions, 16 pages) \\

Chin - Chapters 1 to 5

\textbf{1 General probability theory} (47 questions) \\
\textbf{2 Wiener process} (39 questions) \\
\textbf{3 Stochastic differential equations} (53 questions) \\
\textbf{4 Change of measure} (36 questions) \\
\textbf{5 Poisson process} (47 questions)

\newpage

\begin{center}
\textbf{2. Interest rate models and hedging}
\end{center}

Hirsa \& Neftci - An Introduction to the Mathematics of Financial Derivatives  - Chapters 16 to 20

\textbf{16 New results and tools for interest sensitive securities} (4 questions, 8 pages) \\
\textbf{17 Arbitrage theorem in a new setting} (10 questions, 24 pages) \\
\textbf{18 Modelling term structure and related concepts} (4 questions, 14 pages) \\
\textbf{19 Classical and HJM approach to fixed income} (6 questions, 17 pages) \\
\textbf{20 Classical PDE analysis for interest rate derivatives} (4 questions, 12 pages)

Veronesi - Fixed Income Securities: Valuation, Risk and Risk Management - Chapters 14 to 21 minus 17 plus 22.1 to 22.4

\textbf{14 Interest rate models in continuous time} (8 questions, 32 pages) \\
\textbf{15 No arbitrage and the pricing of interest rate securities} (13 questions, 32 pages) \\
\textbf{16 Dynamic hedging and relative value trades} (7 questions, 30 pages) \\
\textbf{18 The risk and return of interest rate securities} (5 questions, 24 pages) \\
\textbf{19 No arbitrage models and standard derivatives} (5 questions, 34 pages) \\
\textbf{20 The market model and options volatility dynamics} (6 questions, 22 pages) \\
\textbf{21 Forward risk neutral pricing and the LIBOR market model} (9 questions, 38 pages) \\
\textbf{22 Multifactor models} (6 questions, 44 pages) 

\begin{center}
\textbf{3. Equity option pricing and hedging}
\end{center}

Derman \& Miller - The Volatility Smile - Chapters 1 to 11, 14, and 17 to 19

\textbf{01 Overview} \\
\textbf{02 The principle of replication} (3 questions, 24 pages) \\
\textbf{03 Static and dynamic replication} (7 questions, 19 pages) \\
\textbf{04 Variance swaps - a lesson in replication} (6 questions, 28 pages) \\
\textbf{05 The P\&L of hedged option strategies in a Black-Scholes-Merton world} (4 questions, 19 pages) \\
\textbf{06 The effect of discrete hedging on P\&L} (3 questions, 12 pages) \\
\textbf{07 The effect of transaction costs on P\&L} (3 questions, 13 pages) \\
\textbf{08 The smile - stylised facts and their interpretation} (6 questions, 22 pages) \\
\textbf{09 No-arbitrage bounds on the smile} (3 questions, 10 pages) \\
\textbf{10 A survey of smile models} (1 question, 11 pages) \\

\textbf{11 Implied distributions and static replication} (4 questions, 27 pages) \\
\textbf{14 Local volatility models} (4 questions, 16 pages) \\
\textbf{17 Some final remarks on local volatility models} (0 questions, 6 pages) \\
\textbf{18 Patterns of volatility change} (3 questions, 9 pages) \\
\textbf{19 Introducing stochastic volatility models} (4 questions, 17 pages)

\begin{center}
\textbf{4. Applications}
\end{center}

Bunch of readings from SOA themselves

\newpage

\begin{center}
\textbf{Week 1}
\end{center}

\textbf{Monday} \hfill \textbf{20th January 2025}

ALTAM Q1 \\
QFIQF Fall 2020 Q1, 2, 3

\textbf{Tuesday} \hfill \textbf{21st January 2025}

ALTAM Q2, 3 \\
QFIQF Fall 2020, Q4, 5, 6

\textbf{Wednesday} \hfill \textbf{22nd January 2025}

ALTAM Q4 \\
QFIQF Fall 2020, Q7, 8, 9

\textbf{Thursday} \hfill \textbf{23rd January 2025}

ALTAM Q5, 6 \\
QFIQF Fall 2020 Q10, 11, 12

\textbf{Friday} \hfill \textbf{24th January 2025}

ALTAM Q7 \\
QFIQF Fall 2020 Q13, 14, 15

\begin{center}
\textbf{Week 2}
\end{center}

\textbf{Monday} \hfill \textbf{27th January 2025}

ALTAM Q8 \\
QFIQF Spring 2021 Q1, 2, 3

\textbf{Tuesday} \hfill \textbf{28th January 2025}

ALTAM Q9, 10 \\
QFIQF Spring 2021 Q4, 5, 6

\textbf{Wednesday} \hfill \textbf{29th January 2025}

ALTAM Q11 \\
QFIQF Spring 2021 Q7, 8, 9

\textbf{Thursday} \hfill \textbf{30th January 2025}

ALTAM Q12, 13 \\
QFIQF Spring 2021 Q10, 11, 12

\textbf{Friday} \hfill \textbf{31st January 2025}

ALTAM Q14 \\
QFIQF Spring 2021 Q13, 14, 15

\begin{center}
\textbf{Week 3}
\end{center}

\textbf{Monday} \hfill \textbf{3rd February 2025}

REGISTRATION DAY (and also Emma's birthday and Greg says she wants sushi...)

ALTAM Q15 \\
QFIQF Fall 2021 Q1, 2, 3

\textbf{Tuesday} \hfill \textbf{4th February 2025}

ALTAM Q16, 17 \\
QFIQF Fall 2021 Q4, 5, 6

\textbf{Wednesday} \hfill \textbf{5th February 2025}

ALTAM Q18 \\
QFIQF Fall 2021 Q7, 8, 9

\textbf{Thursday} \hfill \textbf{6th February 2025}

ALTAM Q19, 20 \\
QFIQF Fall 2021 Q10, 11, 12

\textbf{Friday} \hfill \textbf{7th February 2025}

ALTAM Q21 \\
QFIQF Fall 2021 Q13, 14, 15

\begin{center}
\textbf{Week 4}
\end{center}

\textbf{Monday} \hfill \textbf{10th February 2025}

ALTAM Q22 \\
QFIQF Spring 2022 Q1, 2, 3

\textbf{Tuesday} \hfill \textbf{11th February 2025}

ALTAM Q23, 24 \\
QFIQF Spring 2022 Q4, 5, 6

\textbf{Wednesday} \hfill \textbf{12th February 2025}

ALTAM Q25 \\
QFIQF Spring 2022 Q7, 8 ,9

\textbf{Thursday} \hfill \textbf{13th February 2025}

ALTAM Q26, 27 \\
QFIQF Spring 2022 Q10, 11, 12

\textbf{Friday} \hfill \textbf{14th February 2025}

ALTAM Q28 \\
QFIQF Spring 2022 Q13, 14, 15

\begin{center}
\textbf{Week 5}
\end{center}

\textbf{Monday} \hfill \textbf{17th February 2025}

ALTAM Q29 \\
QFIQF Fall 2022 Q1, 2, 3

\textbf{Tuesday} \hfill \textbf{18th February 2025}

ALTAM Q30, 31 \\
QFIQF Fall 2022 Q4, 5, 6

\textbf{Wednesday} \hfill \textbf{19th February 2025}

ALTAM Q32 \\
QFIQF Fall 2022 Q7, 8, 9

\textbf{Thursday} \hfill \textbf{20th February 2025}

ALTAM Q33, 34 \\
QFIQF Fall 2022 Q10, 11, 12

\textbf{Friday} \hfill \textbf{21st February 2025}

ALTAM Q35 \\
QFIQF Fall 2022 Q13, 14, 15

\begin{center}
\textbf{Week 6}
\end{center}

\textbf{Monday} \hfill \textbf{24th February 2025}

ALTAM Q36 \\
QFIQF Spring 2023 Q1, 2, 3

\textbf{Tuesday} \hfill \textbf{25th February 2025}

ALTAM Q37, 38 \\
QFIQF Spring 2023 Q4, 5

\textbf{Wednesday} \hfill \textbf{26th February 2025}

ALTAM Q39 \\
QFIQF Spring 2023 Q6, 7

\textbf{Thursday} \hfill \textbf{27th February 2025}

ALTAM Q40, 41 \\
QFIQF Spring 2023 Q8, 9

\textbf{Friday} \hfill \textbf{28th February 2025}

AIM TO PRELIM ON THIS DAY

ALTAM Q42 \\
QFIQF Spring 2023 Q10, 11

\begin{center}
\textbf{Week 7}
\end{center}

\textbf{Monday} \hfill \textbf{3rd March 2025}

ALTAM Q43 \\
QFIQF Fall 2023 Q1, 2

\textbf{Tuesday} \hfill \textbf{4th March 2025}

ALTAM Q44, 45 \\
QFIQF Fall 2023 Q3, 4

\textbf{Wednesday} \hfill \textbf{5th March 2025}

ALTAM Q46 \\
QFIQF Fall 2023 Q5, 6

\textbf{Thursday} \hfill \textbf{6th March 2025}

ALTAM Q47, 48 \\
QFIQF Fall 2023 Q7, 8

\textbf{Friday} \hfill \textbf{7th March 2025}

ALTAM Q49 \\
QFIQF Fall 2023 Q9, 10

\begin{center}
\textbf{Week 8}
\end{center}

\textbf{Monday} \hfill \textbf{10th March 2025}

\textbf{Tuesday} \hfill \textbf{11th March 2025}

FLYING BACK TO SINGAPORE TONIGHT 9.30PM

\textbf{Wednesday} \hfill \textbf{12th March 2025}

\textbf{Thursday} \hfill \textbf{13th March 2025}

\textbf{Friday} \hfill \textbf{14th March 2025}

\end{document}