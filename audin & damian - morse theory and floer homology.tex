\documentclass[hidelinks, 12pt]{article} 
\usepackage{geometry}   \geometry{letterpaper}  
\usepackage{color}

\usepackage[parfill]{parskip}   
\usepackage{graphicx}	
	
\addtolength{\oddsidemargin}{-0.8in}
\addtolength{\evensidemargin}{-0.8in}
\addtolength{\textwidth}{1.6in}
\addtolength{\topmargin}{-.5in}
\addtolength{\textheight}{1in}	
	
\usepackage{amssymb}
\usepackage{amsmath}
\usepackage{commath}

\DeclareMathOperator{\Crit}{Crit}
\DeclareMathOperator{\Id}{Id}
\DeclareMathOperator{\im}{im}

\usepackage{amsthm}

\newtheoremstyle{mydefstyle}
    {6pt}
    {3pt}
    {}
    {}
    {\bfseries}
    {}
    { }
    {\thmname{#1} \normalfont{\thmnote{(#3)}\addcontentsline{toc}{subsubsection}{\bf{#1} \normalfont{(#3)}}}}
    
\theoremstyle{mydefstyle}
\newtheorem{definition}{Definition}
\newtheorem{example}{Example}

\newtheoremstyle{mythmstyle}
    {6pt}
    {3pt}
    {}
    {}
    {\bfseries}
    {}
    { }
    {\thmname{#1} \thmnumber{#2} \normalfont{\thmnote{(#3)}\addcontentsline{toc}{subsubsection}{\bf{#1 #2} \normalfont{(#3)}}}}
    
\theoremstyle{mythmstyle} 
\newcounter{prop}
\newtheorem{proposition}[prop]{Proposition}
\newtheorem{theorem}[prop]{Theorem}
\newtheorem{corollary}[prop]{Corollary}
\newtheorem{lemma}[prop]{Lemma}

\usepackage{tikz-cd}
\usepackage{bm}
\usepackage[shortlabels]{enumitem}












\title{Morse Theory and Floer Homology}
\date{}

\begin{document}
%\maketitle
\pagecolor{white}
%\tableofcontents

\textbf{4.4 Euler Characteristic, Poincar{\'e} Polynomial}

The number of critical points modulo 2 of a Morse function depends only on the manifold and not on the function.

More precisely, we have
\begin{gather*}
\#\Crit(f) \ge \sum_{k = 0}^n \left( \dim \ker \partial_k - \dim \im \partial_{k+1} \right) = \sum_{k = 0}^n \dim HM_k(V; \mathbb{Z}/2)
\end{gather*}

If we define $c_k(f)$ to be the number of critical points of index $k$ of the Morse function $f$ and let $\beta_k = \dim HM_k(V; \mathbb{Z}/2)$ (the $k$th Betti number of $V$), then the proposition can be written as
\begin{gather*}
c_k(f) \ge \beta_k, \qquad k \ge 0
\end{gather*}
a series of inequalities known as the Morse inequalities. 

\textbf{6.1 The Arnold Conjecture}

Let $W$ be a compact symplectic manifold and let
\begin{gather*}
H : W \times \mathbb{R} \to \mathbb{R}
\end{gather*}
be a time-dependent Hamiltonian. Suppose that the solutions of period 1 of the associated Hamiltonian system are nondegenerate. Then their number is greater than or equal to the sum
\begin{gather*}
\sum_i \dim HM_i(W; \mathbb{Z}/2)
\end{gather*}

\end{document}