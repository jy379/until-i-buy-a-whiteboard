\documentclass[hidelinks, 12pt]{article} 
\usepackage{geometry}   \geometry{letterpaper}  
\usepackage{color}

\usepackage[parfill]{parskip}   
\usepackage{graphicx}	
	
\addtolength{\oddsidemargin}{-0.8in}
\addtolength{\evensidemargin}{-0.8in}
\addtolength{\textwidth}{1.6in}
\addtolength{\topmargin}{-.5in}
\addtolength{\textheight}{1in}	
	
\usepackage{amssymb}
\usepackage{amsmath}
\usepackage{commath}

\DeclareMathOperator{\Id}{Id}
\DeclareMathOperator{\Proj}{Proj}

\usepackage{amsthm}

\newtheoremstyle{mydefstyle}
    {6pt}
    {3pt}
    {}
    {}
    {\bfseries}
    {}
    { }
    {\thmname{#1} \normalfont{\thmnote{(#3)}\addcontentsline{toc}{subsubsection}{\bf{#1} \normalfont{(#3)}}}}
    
\theoremstyle{mydefstyle}
\newtheorem{definition}{Definition}
\newtheorem{example}{Example}

\newtheoremstyle{mythmstyle}
    {6pt}
    {3pt}
    {}
    {}
    {\bfseries}
    {}
    { }
    {\thmname{#1} \thmnumber{#2} \normalfont{\thmnote{(#3)}\addcontentsline{toc}{subsubsection}{\bf{#1 #2} \normalfont{(#3)}}}}
    
\theoremstyle{mythmstyle} 
\newcounter{prop}
\newtheorem{proposition}[prop]{Proposition}
\newtheorem{theorem}[prop]{Theorem}
\newtheorem{corollary}[prop]{Corollary}
\newtheorem{lemma}[prop]{Lemma}

\usepackage{tikz-cd}
\usepackage{bm}
\usepackage[shortlabels]{enumitem}












\title{Algebraic Geometry}
\date{}

\begin{document}
%\maketitle
\pagecolor{white}
%\tableofcontents

\textbf{Problem III.4.7} Let $X$ be a subscheme of $\mathbb{P}^2_k$ defined by a single homogeneous equation $f(x_0, x_1, x_2) = 0$ of degree $d$. (Do not assume $f$ is irreducible.) Assume that $(1, 0, 0)$ is not on $X$. Then show that $X$ can be covered by the two open affine subsets $U = X \cap \{x_1 \ne 0\}$ and $V = X \cap \{x_2 \ne 0\}$. Now calculate the Cech complex
\begin{gather*}
\Gamma(U, \mathcal{O}_X) \oplus \Gamma(V, \mathcal{O}_X) \to \Gamma(U \cap V, \mathcal{O}_X)
\end{gather*}
explicitly, and thus show that
\begin{gather*}
\dim H^0(X, \mathcal{O}_X) = 1, \qquad \dim H^1(X, \mathcal{O}_X) = \frac{(d-1)(d-2)}{2}
\end{gather*}

\textbf{III.5 The Cohomology of Projective Space}

\textbf{Theorem 5.1}  Let $A$ be a noetherian ring, let $S = A[x_0, \dots, x_r]$, and let $X = \Proj S$ be the projective space $\mathbb{P}^r_A$ over $A$. Let $\mathcal{O}_X(1)$ be the twisting sheaf of Serre (II, \S 5). For any sheaf of $\mathcal{O}_X$-modules $\mathcal{F}$, we denote by $\Gamma_*(\mathcal{F})$ the graded $S$-module $\oplus_{n\in\mathbb{Z}}\Gamma(X, \mathcal{F}(n))$ (see II, \S 5). Then
\begin{enumerate}[(\alph*)]
\item the natural map $S \to \Gamma_*(\mathcal{O}_X) = \oplus_{n\in\mathbb{Z}}H^0(X, \mathcal{O}_X(n))$ is an isomorphism of graded $S$-modules
\item $H^i(X, \mathcal{O}_X(n)) = 0$ for $0 < i < r$ and all $n \in \mathbb{Z}$
\item $H^r(X, \mathcal{O}_X(-r-1)) \simeq A$
\item the natural map
\begin{gather*}
H^0(X, \mathcal{O}_X(n)) \times H^r(X, \mathcal{O}_X(-n-r-1)) \to H^r(X, \mathcal{O}_X(-r-1)) \simeq A
\end{gather*}
is a perfect pairing of finitely generated free $A$-modules, for each $n \in \mathbb{Z}$.
\end{enumerate}

\textbf{Problem III.5.3} Let $X$ be a projective scheme of dimension $r$ over a field $k$. We define the arithmetic genus $p_a$ of $X$ by
\begin{gather*}
p_a(X) = (-1)^r(\chi(\mathcal{O}_X)-1)
\end{gather*}
Note that it depends only on $X$, not on any projective embedding. 
\begin{enumerate}[(\alph*)]
\item If $X$ is integral, and $k$ algebraically closed, show that $H^0(X, \mathcal{O}_X) \simeq k$, so that
\begin{gather*}
p_a(X) = \sum_{i=0}^{r-1} (-1)^i \dim_k H^{r-i}(X, \mathcal{O}_X)
\end{gather*}
In particular, if $X$ is a curve, we have
\begin{gather*}
p_a(X) = \dim_k H^1(X, \mathcal{O}_X)
\end{gather*}
Hint: Use (I, 3.4)
\item If $X$ is a closed subvariety of $\mathbb{P}^r_k$, show that this $p_a(X)$ coincides with the one defined in (I, Ex. 7.2), which apparently depended on the projective embedding. 
\item If $X$ is a nonsingular projective curve over an algebraically closed field $k$, show that $p_a(X)$ is in fact a birational invariant. Conclude that a nonsingular plane curve of degree $d \ge 3$ is not rational. 
\end{enumerate}

\end{document}