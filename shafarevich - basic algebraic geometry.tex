\documentclass[hidelinks, 12pt]{article} 
\usepackage{geometry}   \geometry{letterpaper}  
\usepackage{color}

\usepackage[parfill]{parskip}   
\usepackage{graphicx}	
	
\addtolength{\oddsidemargin}{-0.8in}
\addtolength{\evensidemargin}{-0.8in}
\addtolength{\textwidth}{1.6in}
\addtolength{\topmargin}{-.5in}
\addtolength{\textheight}{1in}	
	
\usepackage{amssymb}
\usepackage{amsmath}
\usepackage{commath}

\DeclareMathOperator{\Id}{Id}

\usepackage{amsthm}

\newtheoremstyle{mydefstyle}
    {6pt}
    {3pt}
    {}
    {}
    {\bfseries}
    {}
    { }
    {\thmname{#1} \normalfont{\thmnote{(#3)}\addcontentsline{toc}{subsubsection}{\bf{#1} \normalfont{(#3)}}}}
    
\theoremstyle{mydefstyle}
\newtheorem{definition}{Definition}
\newtheorem{example}{Example}

\newtheoremstyle{mythmstyle}
    {6pt}
    {3pt}
    {}
    {}
    {\bfseries}
    {}
    { }
    {\thmname{#1} \thmnumber{#2} \normalfont{\thmnote{(#3)}\addcontentsline{toc}{subsubsection}{\bf{#1 #2} \normalfont{(#3)}}}}
    
\theoremstyle{mythmstyle} 
\newcounter{prop}
\newtheorem{proposition}[prop]{Proposition}
\newtheorem{theorem}[prop]{Theorem}
\newtheorem{corollary}[prop]{Corollary}
\newtheorem{lemma}[prop]{Lemma}

\usepackage{tikz-cd}
\usepackage{bm}
\usepackage[shortlabels]{enumitem}












\title{Basic Algebraic Geometry}
\date{}

\begin{document}
%\maketitle
\pagecolor{white}
%\tableofcontents

\textbf{1.1 Plane Curves}

An algebraic plane curve is a curve consisting of the points of the plane whose coordinates $x, y$ satisfy an equation
\begin{gather*}
f(x, y) = 0 \tag{1}
\end{gather*}
where $f$ is a nonconstant polynomial. We fix a field $k$ and assume that the coordinates $x, y$ of points and the coefficients of $f$ are elements of $k$. 

\textbf{Lemma} Let $k$ be an arbitrary field, $f \in k[x, y]$ an irreducible polynomial, and $g \in k[x, y]$ an arbitrary polynomial. If $g$ is not divisible by $f$ then the system of equations $f(x, y) = g(x, y) = 0$ has only a finite number of solutions.

An algebraically closed field $k$ is infinite, and if $f$ is not a constant, the curve with equation $f(x, y) = 0$ has infinitely many points. Because of this, it follows from the lemma that an irreducible polynomial $f(x, y)$ is uniquely determined, up to a constant multiple, by the curve $f(x, y) = 0$. 

\textbf{1.2 Rational Curves}

We say that an irreducible algebraic curve $X$ defined by $f(x, y) = 0$ is rational if there exist two rational functions $\varphi(t)$ and $\psi(t)$, at least one nonconstant, such that
\begin{gather*}
f(\varphi(t), \psi(t)) \equiv 0 \tag{3}
\end{gather*}
as an identity in $t$. 

\textbf{1.3 Relation with Field Theory}

Let $X$ be the irreducible curve given by 1.1, (1). Consider rational functions $u(x, y) = p(x, y)/q(x, y)$, where $p$ and $q$ are polynomials in $k$ such that the denominator $q(x, y)$ is not divisible by $f(x, y)$. We say that such a function $u(x, y)$ is a \emph{rational function} defined on $X$; and two rational functions $p(x, y)/q(x, y)$ and $p_1(x, y)/q_1(x, y)$ defined on $X$ are equal on $X$ if the polynomial $p(x, y)q_1(x, y) - q(x, y)p_1(x, y)$ is divisible by $f(x, y)$. It is easy to check that rational functions on $X$, up to equality on $X$, form a field. This field is called the function field or field of rational functions of $X$, and denoted by $k(X)$.

Let $X$ be a rational curve. By L{\"u}roth's theorem, the field $k(X)$ is isomorphic to the field of rational functions $k(t)$. Suppose that this isomorphism takes $x$ to $\varphi(t)$ and $y$ to $\psi(t)$. This gives the parametrisation $x = \varphi(t)$, $y = \psi(t)$ of $X$.

\textbf{Proposition} The parametrisation $x = \varphi(t)$, $y = \psi(t)$ has the following properties:
\begin{enumerate}[(i)]
\item Except possibly for a finite number of points, any $(x_0, y_0) \in X$ has a representation $(x_0, y_0) = (\varphi(t_0), \psi(t_0))$ for some $t_0$.
\item Except possibly for a finite number of points, this representation is unique. 
\end{enumerate}

\textbf{1.4 Rational Maps}

\end{document}